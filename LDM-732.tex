\documentclass[DM,lsstdraft,STS,toc]{lsstdoc}
\usepackage{enumitem}
\usepackage{booktabs}
\usepackage{arydshln}
\usepackage{afterpage}
\usepackage{pdflscape}
\usepackage{graphicx}
\usepackage{lipsum}


\input{meta.tex}

%\input{aglossary.tex}
%\makeglossaries

\DeclareRobustCommand{\cmdkey}{\raisebox{-.035em}{\includegraphics[height=.75em]{figures/cmdkey}}}

\begin{document}

\providecommand{\tightlist}{%
  \setlength{\itemsep}{0pt}\setlength{\parskip}{0pt}}

\def\product{LSST Data Management}

\setDocCompact{true}

\title[Network Verification Document]{Vera C. Rubin Observatory Network Verification Document}

\author{Jeff Kantor}
\setDocRef{\lsstDocType-\lsstDocNum}
\setDocDate{\vcsdate}

\setDocAbstract {
The Vera C Rubin Observatory Network Verification Document (VNVD) and associated JIRA V\&V
Project define the flow-down of specifications from higher level documents to the LSST
Observatory Network (as defined in LSE-78 LSST Observatory Network), and the methods and
resources that will be used to verify that the networks have met the specifications satisfactory
for accepting the Summit Network into DM Subsystem Integration Test (DMSSIT) and LSST
System Integration Test (SIT)
}

% Most recent last
\setDocChangeRecord{%
	\addtohist{}{2020-01-28}{First draft}{J.~Kantor}
	\addtohist{}{2020-02-21}{Document ready for CCB approval}{J.~Kantor}
}

\setDocUpstreamLocation{\url{https://github.com/lsst/ldm-732}}
\setDocUpstreamVersion{\vcsrevision}

\maketitle


\section{Introduction}\label{sec:intro}


\subsection{Scope}\label{sec:scope}

This plan governs only tests of the network infrastructure, not the applications and services that
use the network. To be specific, this plan governs tests of the network only up to ISO OSI Layer
\url{https://en.wikipedia.org/wiki/OSI_model}.

As such, all of the tests governed by this plan and defined in the 
\href{https://jira.lsstcorp.org/projects/LVV/}{LSST Verification and Validation JIRA Project (LVV)} 
are defined as Lower Level (LL) in the System Engineering test hierarchy. LL
corresponds to Data Management Subsystem Integration. Where appropriate, additional
comments regarding Same Level (SL) which corresponds to LSST System Integration, and Higher
Level (HL) which corresponds to LSST Commissioning, are called out in the Verification
Elements.

Note that significant testing of the networks occurs prior to subsystem and system integration,
i.e. prior to verification, as documented in document-14789 LSST LHN End-to-End Plan and
associated test documentation 
(see \href{https://docushare.lsstcorp.org/docushare/dsweb/View/Collection-3758}{Collection-3758}).

Finally, note that one significant network, the Summit Network, is not a DM deliverable and as
such is not contained within this plan. As Telescope and Site deliverable, the Summit Network
is covered by the Telescope and Site V\&V plans.

\subsection{Specification Flow-down}\label{sec:sepcflowd}

\subsubsection{Data Management Subsystem Requirements Flow-down}\label{sec:dmreqflowd}

The Data Management Subsystem Requirements (LSE-61, aka DMSR) drive the LSST Observatory Network Design 
for all segments except the Summit Network (see above).  The DMSR sections that directly drive the 
VNVD are listed here for convenience. These DMSR sections contain traceable network requirements as 
documented in the LSST V\&V JIRA Project (LVV) Verification Elements:

\begin{itemize}
\item 1.2.1 Nightly Data Accessible Within 24 hrs
\item 2.6.3 Transient Alert Distribution
\item 2.6.8 Solar System Objects Available within 24 hours
\item 2.8.1 Timely Publication of Level 2 Data Releases
\item 4.4 Summit to Base
\item 4.6 Base to Archive
\item 4.8 Archive to Data Access Center
\end{itemize}

\subsubsection{Observatory System Specifications Flow-down}\label{sec:ossflowd}

Note that the Observatory System Specifications (LSE-30, aka OSS) also include general requirements 
on security, disaster recovery, physical environment (including seismic activity), and shipping 
which are flowed down to the subsystems, and while they apply to all subsystems, including the networks, 
they will be tested and verified in the Telescope and Data Management Subsystem Integration Tests and in 
the LSST Commissioning Phase, as part of the LSST System Integration Test.  Those requirements are excluded 
from this specification and the associated verification matrix, as they will be addressed in the applicable plans.

\subsection{LSST Verification and Validation JIRA Project (LVV)}\label{sec:lvv}

The LSST Verification and Validation JIRA Project lists the specifications within or derived from, 
and traceable to, the DMSR specifications, in Verification Elements that also specify the methods 
to be used to verify, the responsible parties, and additional notes regarding verification, per the 
\citeds{LSE-160} LSST Verification and Validation Process. The Verification Elements then have one or more 
Test Cases associated with them that describe the implementation of the verification activities in terms 
of specific tests to be executed.  Those Test Cases are then scheduled via Test Plans and Campaigns, 
and executed with results reported in Test Cycles/

\subsection{Verification and Validation Schedule and Resources}\label{sec:schedule}

The schedule and resources required for network verification are defined in the LSST 
Project Management Control System (PMCS).  They are covered by the final integration test activities in 
the WBS elements 02C.08.03 Long-Haul Networks.  In each Verification Element, a cross-reference to the 
ID of the appropriate predecessor PMCS activities is provided in the pre-conditions field.


\subsection{Applicable Documents}
\label{sec:docs}

\begin{tabular}[htb]{l l}
\citeds{LSE-61}  & LSST DM Subsystem Requirements \\
\citeds{LSE-78}  & LSST Observatory Network Design \\
\citeds{LSE-160} & Verification and Validation Process \\
\end{tabular}


\newpage
%%%%%%%%%%%%%%%%%%%%%%%%%%%%%%%%%%%%%%%%%%%%%%%%%%%%%%%%%%%%%%%%%%%%%%%%%%%%%%%%%%%%%%%%%%%%%%%
% generated from JIRA project LVV
% using template at /usr/share/miniconda/envs/docsteady-old/lib/python3.7/site-packages/docsteady/templates/ve.latex.jinja2.
% using docsteady version 2.0
% Please do not edit -- update information in Jira instead
%%%%%%%%%%%%%%%%%%%%%%%%%%%%%%%%%%%%%%%%%%%%%%%%%%%%%%%%%%%%%%%%%%%%%%%%%%%%%%%%%%%%%%%%%%%%%%%

\section{ DM - NETWORK Verification Elements }
\label{sec:ves}

The following is the list of verification elements defined in the context of the NETWORK
component\footnote{Major product in the subsystem.} of the DM subsystem.


\subsection{[LVV-71] DMS-REQ-0168-V-01: Summit Facility Data Communications }\label{lvv-71}

\begin{longtable}{cccc}
\hline
\textbf{Jira Link} & \textbf{Assignee} & \textbf{Status} & \textbf{Test Cases}\\ \hline
\href{https://jira.lsstcorp.org/browse/LVV-71}{LVV-71} &
Gregory Dubois-Felsmann & Covered &
\begin{tabular}{c}
LVV-T1097 \\
LVV-T2338 \\
\end{tabular}
\\
\hline
\end{longtable}

\textbf{Verification Element Description:} \\
Verify that:

\begin{itemize}
\tightlist
\item
  Summit - Base Network has been properly implemented in Summit and Base
  facilities
\item
  Summit - Base Network is properly integrated with Summit Control
  Network and DAQ/Camera Data Backbone
\end{itemize}

Verify that OCS/DMCS triggers read-out from DAQ and queries EFD. verify
that data from EFD and camera are accepted and transferred to the Summit
DWDM. Requirement does not include data transfer to base
(\href{https://jira.lsstcorp.org/browse/LVV-73}{LVV-73}) or from base to
archive center (\href{https://jira.lsstcorp.org/browse/LVV-81}{LVV-81},
\href{https://jira.lsstcorp.org/browse/LVV-82}{LVV-82},
\href{https://jira.lsstcorp.org/browse/LVV-83}{LVV-83}).

{\footnotesize
\begin{tabular}{p{4cm}p{12cm}}
\hline
\multicolumn{2}{c}{\textbf{Requirement Details}}\\ \hline
Requirement ID & DMS-REQ-0168 \\ \hline
Requirement Priority & 1a \\ \hline
\multicolumn{2}{l}{Requirement Description and Discussion:} \\ \cdashline{1-2}
\end{tabular}

\textbf{Specification:} The DMS shall provide data communications
infrastructure to accept science data and associated metadata read-outs,
and the collection of ancillary and engineering data, for transfer to
the base facility.

\begin{longtable}{p{4cm}p{12cm}}
\hline
Upper Level Requirement &
\begin{tabular}{cl}
OSS-REQ-0002 & The Summit Facility \\
\end{tabular}
\\ \hline
\end{longtable}
}


\subsubsection{Test Cases Summary}
\begin{tabular}{p{3cm}p{2.5cm}p{2.5cm}p{3cm}p{4cm}}
\toprule
\href{https://jira.lsstcorp.org/secure/Tests.jspa\#/testCase/LVV-T1097}{LVV-T1097} & \multicolumn{4}{p{12cm}}{ Verify Summit Facility Network Implementation } \\ \hline
\textbf{Owner} & \textbf{Status} & \textbf{Version} & \textbf{Critical Event} & \textbf{Verification Type} \\ \hline
Jeff Kantor & Draft & 1 & false & Test \\ \hline
\end{tabular}
{\footnotesize
\textbf{Objective:}\\
Verify that data acquired by a AuxTel DAQ can be transferred to Summit
DWDM and loaded in the EFD without problems.
}

\begin{tabular}{p{3cm}p{2.5cm}p{2.5cm}p{3cm}p{4cm}}
\toprule
\href{https://jira.lsstcorp.org/secure/Tests.jspa\#/testCase/LVV-T2338}{LVV-T2338} & \multicolumn{4}{p{12cm}}{ Replicated telemetry data agrees with telemetry produced at the summit } \\ \hline
\textbf{Owner} & \textbf{Status} & \textbf{Version} & \textbf{Critical Event} & \textbf{Verification Type} \\ \hline
Simon Krughoff & Defined & 1 & false & Demonstration \\ \hline
\end{tabular}
{\footnotesize
\textbf{Objective:}\\
Show that telemetry data can be accessed from the replicated EFD.
~Further, show that the values in the replicated database agree with the
values in the summit EFD over a specified time range and set of
topics.\\[2\baselineskip]This test case provides partial coverage of the
requirement DMS-REQ-0168, Summit Facility Data Communications: "The DMS
shall provide data communications infrastructure to accept science data
and associated metadata read-outs, and \textbf{the collection of
ancillary and engineering data}, for transfer to the base facility.", as
adapted to the current design for EFD replication (see
\href{https://dmtn-082.lsst.io}{DMTN-082}).
}

  
\newpage 
\subsection{[LVV-73] DMS-REQ-0171-V-01: Summit to Base Network }\label{lvv-73}

\begin{longtable}{cccc}
\hline
\textbf{Jira Link} & \textbf{Assignee} & \textbf{Status} & \textbf{Test Cases}\\ \hline
\href{https://jira.lsstcorp.org/browse/LVV-73}{LVV-73} &
Leanne Guy & In Verification &
\begin{tabular}{c}
LVV-T1168 \\
LVV-T1612 \\
\end{tabular}
\\
\hline
\end{longtable}

\textbf{Verification Element Description:} \\
This requirement must be tested in sequence and collect performance
metrics (both DAQ and Control sides unless noted):

\begin{enumerate}
\tightlist
\item
  ISO OSI Layer 1 Physical (fibers with test data from OTDR, AURA does
  test)
\item
  ISO OSI Layer 2 Data Link (DWDM equipment, line cards, with test data
  from multi-channel/lightwave/channel analyzer, Installer does test,
  AURA certify)
\item
  ISO Layer 3 minimal (DWDM with 2 x 10 Gbps ethernet port client cards
  with test data from 4 windows test boxes, 2 on each side, Installer
  does test, AURA certify, can repeat as part of \#4 with DAQ)
\item
  ISO Layer 3 full (22 x 10 Gbps ethernet ports on DAQ side with test
  data from DAQ test stand, AURA, Camera DAQ team do test). Transfer
  data between summit and base over uninterrupted 1 day period.
  ~Demonstrate transfer of data at or exceeding rates specified in
  \citeds{LDM-142}.
\end{enumerate}

{\footnotesize
\begin{tabular}{p{4cm}p{12cm}}
\hline
\multicolumn{2}{c}{\textbf{Requirement Details}}\\ \hline
Requirement ID & DMS-REQ-0171 \\ \hline
Requirement Priority & 1a \\ \hline
\multicolumn{2}{l}{Requirement Description and Discussion:} \\ \cdashline{1-2}
\end{tabular}

\textbf{Specification:} The DMS shall provide communications
infrastructure between the Summit Facility and the Base Facility
sufficient to carry scientific data and associated metadata for each
image in no more than time \textbf{summToBaseMaxTransferTime}.

\begin{longtable}{p{4cm}p{12cm}}
\hline
Requirement Parameters & \textbf{summToBaseMaxTransferTime = 2{{[}second{]}}} Maximum time
interval to transfer a full Crosstalk Corrected Exposure and all related
metadata from the Summit Facility to the Base facility. \\ \hline
Upper Level Requirement &
\begin{tabular}{cl}
OSS-REQ-0003 & The Base Facility \\
OSS-REQ-0127 & Level 1 Data Product Availability \\
\end{tabular}
\\ \hline
\end{longtable}
}


\subsubsection{Test Cases Summary}
\begin{tabular}{p{3cm}p{2.5cm}p{2.5cm}p{3cm}p{4cm}}
\toprule
\href{https://jira.lsstcorp.org/secure/Tests.jspa\#/testCase/LVV-T1168}{LVV-T1168} & \multicolumn{4}{p{12cm}}{ Verify Summit - Base Network Integration } \\ \hline
\textbf{Owner} & \textbf{Status} & \textbf{Version} & \textbf{Critical Event} & \textbf{Verification Type} \\ \hline
Jeff Kantor & Approved & 1 & false & Inspection \\ \hline
\end{tabular}
{\footnotesize
\textbf{Objective:}\\
Verify the integration of the summit to base network by demonstrating a
sustained and uninterrupted transfer of data between summit and base
over 1 day period at or exceeding rates specified in \citeds{LDM-142}. Done in 3
phases in collaboration with equipment/installation vendors (see test
procedure).
}

\begin{tabular}{p{3cm}p{2.5cm}p{2.5cm}p{3cm}p{4cm}}
\toprule
\href{https://jira.lsstcorp.org/secure/Tests.jspa\#/testCase/LVV-T1612}{LVV-T1612} & \multicolumn{4}{p{12cm}}{ Verify Summit - Base Network Integration (System Level) } \\ \hline
\textbf{Owner} & \textbf{Status} & \textbf{Version} & \textbf{Critical Event} & \textbf{Verification Type} \\ \hline
Jeff Kantor & Draft & 1 & false & Inspection \\ \hline
\end{tabular}
{\footnotesize
\textbf{Objective:}\\
Verify ISO Layer 3 full (22 x 10 Gbps ethernet ports on DAQ side with
test data from DAQ test stand, AURA, Camera DAQ team do test).
Demonstrate transfer of data at or exceeding rates specified in \citeds{LDM-142}.
}

  
\newpage 
\subsection{[LVV-74] DMS-REQ-0172-V-01: Summit to Base Network Availability }\label{lvv-74}

\begin{longtable}{cccc}
\hline
\textbf{Jira Link} & \textbf{Assignee} & \textbf{Status} & \textbf{Test Cases}\\ \hline
\href{https://jira.lsstcorp.org/browse/LVV-74}{LVV-74} &
Leanne Guy & Covered &
\begin{tabular}{c}
LVV-T185 \\
\end{tabular}
\\
\hline
\end{longtable}

\textbf{Verification Element Description:} \\
This requirement needs the network link to be active for a calculated
amount of time (at least 1 week) without failure. Will require
extrapolating from test and historical data as failures are rare.
Monthly operating statistics will be acquired during commissioning.
Demonstrate transfer of data at or exceeding rates specified in \citeds{LDM-142},
verify achieved average and peak throughput and latency.

{\footnotesize
\begin{tabular}{p{4cm}p{12cm}}
\hline
\multicolumn{2}{c}{\textbf{Requirement Details}}\\ \hline
Requirement ID & DMS-REQ-0172 \\ \hline
Requirement Priority & 1b \\ \hline
\multicolumn{2}{l}{Requirement Description and Discussion:} \\ \cdashline{1-2}
\end{tabular}

\textbf{Specification:} The Summit to Base communications shall be
highly available, with Mean Time Between Failures (MTBF) \textgreater{}
\textbf{summToBaseNetMTBF}.

\begin{longtable}{p{4cm}p{12cm}}
\hline
Requirement Parameters & \textbf{summToBaseNetMTBF = 90{{[}day{]}}} Mean time between failures,
measured over a 1-yr period. \\ \hline
Upper Level Requirement &
\begin{tabular}{cl}
OSS-REQ-0373 & Unscheduled Downtime Subsystem Allocations \\
DMS-REQ-0161 & Optimization of Cost, Reliability and Availability in Order \\
\end{tabular}
\\ \hline
\end{longtable}
}


\subsubsection{Test Cases Summary}
\begin{tabular}{p{3cm}p{2.5cm}p{2.5cm}p{3cm}p{4cm}}
\toprule
\href{https://jira.lsstcorp.org/secure/Tests.jspa\#/testCase/LVV-T185}{LVV-T185} & \multicolumn{4}{p{12cm}}{ Verify implementation of Summit to Base Network Availability } \\ \hline
\textbf{Owner} & \textbf{Status} & \textbf{Version} & \textbf{Critical Event} & \textbf{Verification Type} \\ \hline
Jeff Kantor & Draft & 1 & false & Inspection \\ \hline
\end{tabular}
{\footnotesize
\textbf{Objective:}\\
Verify the availability of Summit to Base Network by demonstrating that
the mean time between failures is less than summToBaseNetMTBF (90 days)
over 1 year.
}

  
\newpage 
\subsection{[LVV-75] DMS-REQ-0173-V-01: Summit to Base Network Reliability }\label{lvv-75}

\begin{longtable}{cccc}
\hline
\textbf{Jira Link} & \textbf{Assignee} & \textbf{Status} & \textbf{Test Cases}\\ \hline
\href{https://jira.lsstcorp.org/browse/LVV-75}{LVV-75} &
Leanne Guy & Covered &
\begin{tabular}{c}
LVV-T186 \\
\end{tabular}
\\
\hline
\end{longtable}

\textbf{Verification Element Description:} \\
Disconnect, reconnect and recover transfer of data between summit and
base. After disconnecting fiber at an intermediate location between
summit and base, demonstrate reconnection and recovery to transfer of
data at or exceeding rates specified in \citeds{LDM-142} within MTTR
specification. ~Network operator will provide MTTR data on links during
commissioning and operations.

{\footnotesize
\begin{tabular}{p{4cm}p{12cm}}
\hline
\multicolumn{2}{c}{\textbf{Requirement Details}}\\ \hline
Requirement ID & DMS-REQ-0173 \\ \hline
Requirement Priority & 1b \\ \hline
\multicolumn{2}{l}{Requirement Description and Discussion:} \\ \cdashline{1-2}
\end{tabular}

\textbf{Specification:} The Summit to Base communications shall be
highly reliable, with Mean Time to Repair (MTTR) \textless{}
\textbf{summToBaseNetMTTR}.

\begin{longtable}{p{4cm}p{12cm}}
\hline
Requirement Parameters & \textbf{summToBaseNetMTTR = 24{{[}hour{]}}} Mean time to repair,
measured over a 1-yr period. \\ \hline
Upper Level Requirement &
\begin{tabular}{cl}
OSS-REQ-0373 & Unscheduled Downtime Subsystem Allocations \\
DMS-REQ-0161 & Optimization of Cost, Reliability and Availability in Order \\
\end{tabular}
\\ \hline
\end{longtable}
}


\subsubsection{Test Cases Summary}
\begin{tabular}{p{3cm}p{2.5cm}p{2.5cm}p{3cm}p{4cm}}
\toprule
\href{https://jira.lsstcorp.org/secure/Tests.jspa\#/testCase/LVV-T186}{LVV-T186} & \multicolumn{4}{p{12cm}}{ Verify implementation of Summit to Base Network Reliability } \\ \hline
\textbf{Owner} & \textbf{Status} & \textbf{Version} & \textbf{Critical Event} & \textbf{Verification Type} \\ \hline
Jeff Kantor & Draft & 1 & false & Demonstration \\ \hline
\end{tabular}
{\footnotesize
\textbf{Objective:}\\
Verify the reliability of the summit to base network by demonstrating
reconnection and recovery to transfer of data at or exceeding rates
specified in \citeds{LDM-142} following a cut in network connection, within MTTR
specification. The network operator will provide MTTR data on links
during commissioning and operations.\\[2\baselineskip]
}

  
\newpage 
\subsection{[LVV-76] DMS-REQ-0174-V-01: Summit to Base Network Secondary Link }\label{lvv-76}

\begin{longtable}{cccc}
\hline
\textbf{Jira Link} & \textbf{Assignee} & \textbf{Status} & \textbf{Test Cases}\\ \hline
\href{https://jira.lsstcorp.org/browse/LVV-76}{LVV-76} &
Leanne Guy & Covered &
\begin{tabular}{c}
LVV-T187 \\
\end{tabular}
\\
\hline
\end{longtable}

\textbf{Verification Element Description:} \\
This requirement is verified by demonstrating use of a secondary
transfer method (redundant fiber network, microwave link, or
transportable medium) between Summit and Base capable of transferring 1
night of raw data (nCalibExpDay + nRawExpNightMax = 450 + 2800 = 3250
exposures) within summToBaseNet2TransMax (72 hours).

{\footnotesize
\begin{tabular}{p{4cm}p{12cm}}
\hline
\multicolumn{2}{c}{\textbf{Requirement Details}}\\ \hline
Requirement ID & DMS-REQ-0174 \\ \hline
Requirement Priority & 1b \\ \hline
\multicolumn{2}{l}{Requirement Description and Discussion:} \\ \cdashline{1-2}
\end{tabular}

\textbf{Specification:} The Summit to Base communications shall provide
at least one secondary link or transport mechanism for minimal
operations support in the event of extended outage. This link may
include redundant fiber optics, microwaves, or transportable media. It
shall be capable of transferring one night's worth of raw data in
\textbf{summToBaseNet2TransMax} or less.

\begin{longtable}{p{4cm}p{12cm}}
\hline
Requirement Parameters & \textbf{summToBaseNet2TransMax = 72{{[}hour{]}}} Maximum time to
transfer one night of data via the network secondary link. \\ \hline
Upper Level Requirement &
\begin{tabular}{cl}
DMS-REQ-0173 & Summit to Base Network Reliability \\
OSS-REQ-0049 & Degraded Operational States \\
DMS-REQ-0172 & Summit to Base Network Availability \\
\end{tabular}
\\ \hline
\end{longtable}
}


\subsubsection{Test Cases Summary}
\begin{tabular}{p{3cm}p{2.5cm}p{2.5cm}p{3cm}p{4cm}}
\toprule
\href{https://jira.lsstcorp.org/secure/Tests.jspa\#/testCase/LVV-T187}{LVV-T187} & \multicolumn{4}{p{12cm}}{ Verify implementation of Summit to Base Network Secondary Link } \\ \hline
\textbf{Owner} & \textbf{Status} & \textbf{Version} & \textbf{Critical Event} & \textbf{Verification Type} \\ \hline
Jeff Kantor & Draft & 1 & false & Test \\ \hline
\end{tabular}
{\footnotesize
\textbf{Objective:}\\
Verify automated fail-over from primary to secondary equipment in Rubin
Observatory DWDM on simulated failure of primary. ~Verify bandwidth
sufficiency on secondary. ~Verify automated recovery to primary
equipment on simulated restoration of primary. ~Repeat for failure of
Rubin Observatory fiber and fail-over to AURA fiber and DWDM.
~Demonstrate use of secondary in ``catch-up'' mode.
}

  
\newpage 
\subsection{[LVV-77] DMS-REQ-0175-V-01: Summit to Base Network Ownership and Operation }\label{lvv-77}

\begin{longtable}{cccc}
\hline
\textbf{Jira Link} & \textbf{Assignee} & \textbf{Status} & \textbf{Test Cases}\\ \hline
\href{https://jira.lsstcorp.org/browse/LVV-77}{LVV-77} &
Leanne Guy & Covered &
\begin{tabular}{c}
LVV-T188 \\
\end{tabular}
\\
\hline
\end{longtable}

\textbf{Verification Element Description:} \\
This requirement is verified by inspecting construction and operations
contracts and Indefeasible Rights to Use (IRUs).

{\footnotesize
\begin{tabular}{p{4cm}p{12cm}}
\hline
\multicolumn{2}{c}{\textbf{Requirement Details}}\\ \hline
Requirement ID & DMS-REQ-0175 \\ \hline
Requirement Priority & 1b \\ \hline
\multicolumn{2}{l}{Requirement Description and Discussion:} \\ \cdashline{1-2}
\end{tabular}

\textbf{Specification:} The Summit to Base communications link shall be
owned and operated by LSST and/or the operations entity to ensure
responsiveness of support.

\begin{longtable}{p{4cm}p{12cm}}
\hline
Upper Level Requirement &
\begin{tabular}{cl}
DMS-REQ-0173 & Summit to Base Network Reliability \\
OSS-REQ-0036 & Local Autonomous Administration of System Sites \\
DMS-REQ-0172 & Summit to Base Network Availability \\
\end{tabular}
\\ \hline
\end{longtable}
}


\subsubsection{Test Cases Summary}
\begin{tabular}{p{3cm}p{2.5cm}p{2.5cm}p{3cm}p{4cm}}
\toprule
\href{https://jira.lsstcorp.org/secure/Tests.jspa\#/testCase/LVV-T188}{LVV-T188} & \multicolumn{4}{p{12cm}}{ Verify implementation of Summit to Base Network Ownership and Operation } \\ \hline
\textbf{Owner} & \textbf{Status} & \textbf{Version} & \textbf{Critical Event} & \textbf{Verification Type} \\ \hline
Jeff Kantor & Draft & 1 & false & Inspection \\ \hline
\end{tabular}
{\footnotesize
\textbf{Objective:}\\
Verify Summit to Base Network Ownership and Operation by LSST and/or the
operations entity by inspection of construction and operations contracts
and Indefeasible Rights.
}

  
\newpage 
\subsection{[LVV-81] DMS-REQ-0180-V-01: Base to Archive Network }\label{lvv-81}

\begin{longtable}{cccc}
\hline
\textbf{Jira Link} & \textbf{Assignee} & \textbf{Status} & \textbf{Test Cases}\\ \hline
\href{https://jira.lsstcorp.org/browse/LVV-81}{LVV-81} &
Leanne Guy & Covered &
\begin{tabular}{c}
LVV-T193 \\
\end{tabular}
\\
\hline
\end{longtable}

\textbf{Verification Element Description:} \\
This requirement is verified by transferring simulated or pre-cursor
image data and meta-data between base and archive over an uninterrupted
1 day period. ~Analyze the network performance and show that data can be
transferred by DAQ within the required time.

{\footnotesize
\begin{tabular}{p{4cm}p{12cm}}
\hline
\multicolumn{2}{c}{\textbf{Requirement Details}}\\ \hline
Requirement ID & DMS-REQ-0180 \\ \hline
Requirement Priority & 1b \\ \hline
\multicolumn{2}{l}{Requirement Description and Discussion:} \\ \cdashline{1-2}
\end{tabular}

\textbf{Specification:} The DMS shall provide communications
infrastructure between the Base Facility and the Archive Center
sufficient to carry scientific data and associated metadata for each
image in no more than time \textbf{baseToArchiveMaxTransferTime}.

\begin{longtable}{p{4cm}p{12cm}}
\hline
Requirement Parameters & \textbf{baseToArchiveMaxTransferTime = 5{{[}second{]}}} Maximum time
interval to transfer a full Crosstalk Corrected Exposure and all related
metadata from the Base Facility to the Archive Center. \\ \hline
Upper Level Requirement &
\begin{tabular}{cl}
OSS-REQ-0053 & Base-Archive Connectivity Loss \\
OSS-REQ-0055 & Base Updating from Archive \\
DMS-REQ-0162 & Pipeline Throughput \\
\end{tabular}
\\ \hline
\end{longtable}
}


\subsubsection{Test Cases Summary}
\begin{tabular}{p{3cm}p{2.5cm}p{2.5cm}p{3cm}p{4cm}}
\toprule
\href{https://jira.lsstcorp.org/secure/Tests.jspa\#/testCase/LVV-T193}{LVV-T193} & \multicolumn{4}{p{12cm}}{ Verify implementation of Base to Archive Network } \\ \hline
\textbf{Owner} & \textbf{Status} & \textbf{Version} & \textbf{Critical Event} & \textbf{Verification Type} \\ \hline
Jeff Kantor & Draft & 1 & false & Test \\ \hline
\end{tabular}
{\footnotesize
\textbf{Objective:}\\
Verify that the data acquired by a DAQ can be transferred within the
required time, i.e. verify that link is capable of transferring image
for prompt processing in oArchiveMaxTransferTime = 5{[}second{]}, i.e.
at or exceeding rates specified in \citeds{LDM-142}.
}

  
\newpage 
\subsection{[LVV-82] DMS-REQ-0181-V-01: Base to Archive Network Availability }\label{lvv-82}

\begin{longtable}{cccc}
\hline
\textbf{Jira Link} & \textbf{Assignee} & \textbf{Status} & \textbf{Test Cases}\\ \hline
\href{https://jira.lsstcorp.org/browse/LVV-82}{LVV-82} &
Leanne Guy & Covered &
\begin{tabular}{c}
LVV-T194 \\
\end{tabular}
\\
\hline
\end{longtable}

\textbf{Verification Element Description:} \\
This requirement is verified by transferring data between base and
archive over uninterrupted 1 week period, modeling to extrapolate to an
annual failure rate, and verifying that is within the requirement.

{\footnotesize
\begin{tabular}{p{4cm}p{12cm}}
\hline
\multicolumn{2}{c}{\textbf{Requirement Details}}\\ \hline
Requirement ID & DMS-REQ-0181 \\ \hline
Requirement Priority & 1b \\ \hline
\multicolumn{2}{l}{Requirement Description and Discussion:} \\ \cdashline{1-2}
\end{tabular}

\textbf{Specification:} The Base to Archive communications shall be
highly available, with MTBF \textgreater{} \textbf{baseToArchNetMTBF}.

\begin{longtable}{p{4cm}p{12cm}}
\hline
Requirement Parameters & \textbf{baseToArchNetMTBF = 180{{[}day{]}}} Mean time between failures,
measured over a 1-yr period. \\ \hline
Upper Level Requirement &
\begin{tabular}{cl}
OSS-REQ-0053 & Base-Archive Connectivity Loss \\
DMS-REQ-0162 & Pipeline Throughput \\
DMS-REQ-0161 & Optimization of Cost, Reliability and Availability in Order \\
\end{tabular}
\\ \hline
\end{longtable}
}


\subsubsection{Test Cases Summary}
\begin{tabular}{p{3cm}p{2.5cm}p{2.5cm}p{3cm}p{4cm}}
\toprule
\href{https://jira.lsstcorp.org/secure/Tests.jspa\#/testCase/LVV-T194}{LVV-T194} & \multicolumn{4}{p{12cm}}{ Verify implementation of Base to Archive Network Availability } \\ \hline
\textbf{Owner} & \textbf{Status} & \textbf{Version} & \textbf{Critical Event} & \textbf{Verification Type} \\ \hline
Jeff Kantor & Draft & 1 & false & Test \\ \hline
\end{tabular}
{\footnotesize
\textbf{Objective:}\\
Verify the availability of the Base to Archive Network communications by
demonstrating that it meets or exceeds a mean time between failures,
measured over a 1-yr period of MTBF \textgreater{} baseToArchNetMTBF
(180{[}day{]})
}

  
\newpage 
\subsection{[LVV-83] DMS-REQ-0182-V-01: Base to Archive Network Reliability }\label{lvv-83}

\begin{longtable}{cccc}
\hline
\textbf{Jira Link} & \textbf{Assignee} & \textbf{Status} & \textbf{Test Cases}\\ \hline
\href{https://jira.lsstcorp.org/browse/LVV-83}{LVV-83} &
Leanne Guy & Covered &
\begin{tabular}{c}
LVV-T195 \\
\end{tabular}
\\
\hline
\end{longtable}

\textbf{Verification Element Description:} \\
Disconnect, reconnect and recover transfer of data between summit and
base, after disconnecting fiber at an intermediate location between base
and archive

{\footnotesize
\begin{tabular}{p{4cm}p{12cm}}
\hline
\multicolumn{2}{c}{\textbf{Requirement Details}}\\ \hline
Requirement ID & DMS-REQ-0182 \\ \hline
Requirement Priority & 1b \\ \hline
\multicolumn{2}{l}{Requirement Description and Discussion:} \\ \cdashline{1-2}
\end{tabular}

\textbf{Specification:} The Base to Archive communications shall be
highly reliable, with MTTR \textless{} \textbf{baseToArchNetMTTR}.

\begin{longtable}{p{4cm}p{12cm}}
\hline
Requirement Parameters & \textbf{baseToArchNetMTTR = 48{{[}hour{]}}} Mean time to repair,
measured over a 1-yr period. \\ \hline
Upper Level Requirement &
\begin{tabular}{cl}
OSS-REQ-0053 & Base-Archive Connectivity Loss \\
DMS-REQ-0161 & Optimization of Cost, Reliability and Availability in Order \\
\end{tabular}
\\ \hline
\end{longtable}
}


\subsubsection{Test Cases Summary}
\begin{tabular}{p{3cm}p{2.5cm}p{2.5cm}p{3cm}p{4cm}}
\toprule
\href{https://jira.lsstcorp.org/secure/Tests.jspa\#/testCase/LVV-T195}{LVV-T195} & \multicolumn{4}{p{12cm}}{ Verify implementation of Base to Archive Network Reliability } \\ \hline
\textbf{Owner} & \textbf{Status} & \textbf{Version} & \textbf{Critical Event} & \textbf{Verification Type} \\ \hline
Jeff Kantor & Draft & 1 & false & Test \\ \hline
\end{tabular}
{\footnotesize
\textbf{Objective:}\\
Verify Base to Archive Network Reliability by demonstrating that the
network can recover from outages within baseToArchNetMTTR =
48{[}hour{]}.
}

  
\newpage 
\subsection{[LVV-84] DMS-REQ-0183-V-01: Base to Archive Network Secondary Link }\label{lvv-84}

\begin{longtable}{cccc}
\hline
\textbf{Jira Link} & \textbf{Assignee} & \textbf{Status} & \textbf{Test Cases}\\ \hline
\href{https://jira.lsstcorp.org/browse/LVV-84}{LVV-84} &
Leanne Guy & Covered &
\begin{tabular}{c}
LVV-T196 \\
\end{tabular}
\\
\hline
\end{longtable}

\textbf{Verification Element Description:} \\
This requirement is verified by disconnecting the primary link, failing
over to the secondary link, reconnecting primary link, and failing back
to primary link, while verifying data is transferred within required
times.

{\footnotesize
\begin{tabular}{p{4cm}p{12cm}}
\hline
\multicolumn{2}{c}{\textbf{Requirement Details}}\\ \hline
Requirement ID & DMS-REQ-0183 \\ \hline
Requirement Priority & 1b \\ \hline
\multicolumn{2}{l}{Requirement Description and Discussion:} \\ \cdashline{1-2}
\end{tabular}

\textbf{Specification:} The Base to Archive communications shall provide
a secondary link or transport mechanism (e.g. protected circuit) for
operations support and ``catch up'' in the event of extended outage
which is capable of transferring data at least the same rate as the
required minimum capacity of the primary link.

\begin{longtable}{p{4cm}p{12cm}}
\hline
Upper Level Requirement &
\begin{tabular}{cl}
DMS-REQ-0181 & Base to Archive Network Availability \\
DMS-REQ-0182 & Base to Archive Network Reliability \\
OSS-REQ-0049 & Degraded Operational States \\
\end{tabular}
\\ \hline
\end{longtable}
}


\subsubsection{Test Cases Summary}
\begin{tabular}{p{3cm}p{2.5cm}p{2.5cm}p{3cm}p{4cm}}
\toprule
\href{https://jira.lsstcorp.org/secure/Tests.jspa\#/testCase/LVV-T196}{LVV-T196} & \multicolumn{4}{p{12cm}}{ Verify implementation of Base to Archive Network Secondary Link } \\ \hline
\textbf{Owner} & \textbf{Status} & \textbf{Version} & \textbf{Critical Event} & \textbf{Verification Type} \\ \hline
Jeff Kantor & Draft & 1 & false & Test \\ \hline
\end{tabular}
{\footnotesize
\textbf{Objective:}\\
Verify Base to Archive Network Secondary Link failover and capacity, and
subsequent recovery primary. Demonstrate the use of the secondary path
in ``catch-up'' mode.
}

  
\newpage 
\subsection{[LVV-88] DMS-REQ-0188-V-01: Archive to Data Access Center Network }\label{lvv-88}

\begin{longtable}{cccc}
\hline
\textbf{Jira Link} & \textbf{Assignee} & \textbf{Status} & \textbf{Test Cases}\\ \hline
\href{https://jira.lsstcorp.org/browse/LVV-88}{LVV-88} &
Leanne Guy & Covered &
\begin{tabular}{c}
LVV-T200 \\
\end{tabular}
\\
\hline
\end{longtable}

\textbf{Verification Element Description:} \\
This requirement is verified by transferring data between archive and
both DACs over uninterrupted 1 day period, analyzing the network
performance, and verifying that data can be transferred within the
required time.

{\footnotesize
\begin{tabular}{p{4cm}p{12cm}}
\hline
\multicolumn{2}{c}{\textbf{Requirement Details}}\\ \hline
Requirement ID & DMS-REQ-0188 \\ \hline
Requirement Priority & 1b \\ \hline
\multicolumn{2}{l}{Requirement Description and Discussion:} \\ \cdashline{1-2}
\end{tabular}

\textbf{Specification:} The DMS shall provide communications
infrastructure between the Archive Center and Data Access Centers
sufficient to carry scientific data and associated metadata in support
of community and EPO access. Aggregate bandwidth for data transfers from
the Archive Center to Data Centers shall be at least
\textbf{archToDacBandwidth}.

\begin{longtable}{p{4cm}p{12cm}}
\hline
Requirement Parameters & \textbf{archToDacBandwidth = 10000{{[}megabit per second{]}}} Aggregate
bandwidth capacity for data transfers between the Archive and Data
Access Centers. \\ \hline
Upper Level Requirement &
\begin{tabular}{cl}
\end{tabular}
\\ \hline
\end{longtable}
}


\subsubsection{Test Cases Summary}
\begin{tabular}{p{3cm}p{2.5cm}p{2.5cm}p{3cm}p{4cm}}
\toprule
\href{https://jira.lsstcorp.org/secure/Tests.jspa\#/testCase/LVV-T200}{LVV-T200} & \multicolumn{4}{p{12cm}}{ Verify implementation of Archive to Data Access Center Network } \\ \hline
\textbf{Owner} & \textbf{Status} & \textbf{Version} & \textbf{Critical Event} & \textbf{Verification Type} \\ \hline
Jeff Kantor & Draft & 1 & false & Test \\ \hline
\end{tabular}
{\footnotesize
\textbf{Objective:}\\
Verify archiving of data to Data Access Center Network at or exceeding
rates specified in \citeds{LDM-142}, i.e at archToDacBandwidth = 10000{[}megabit
per second{]}.
}

  
\newpage 
\subsection{[LVV-89] DMS-REQ-0189-V-01: Archive to Data Access Center Network Availability }\label{lvv-89}

\begin{longtable}{cccc}
\hline
\textbf{Jira Link} & \textbf{Assignee} & \textbf{Status} & \textbf{Test Cases}\\ \hline
\href{https://jira.lsstcorp.org/browse/LVV-89}{LVV-89} &
Leanne Guy & Covered &
\begin{tabular}{c}
LVV-T201 \\
\end{tabular}
\\
\hline
\end{longtable}

\textbf{Verification Element Description:} \\
This requirement needs the network link to be active for a calculated
amount of time (at least 1 week) without failure. This will require
modeling as failures are rare, so an annual MTBF will be estimated from
test results.

{\footnotesize
\begin{tabular}{p{4cm}p{12cm}}
\hline
\multicolumn{2}{c}{\textbf{Requirement Details}}\\ \hline
Requirement ID & DMS-REQ-0189 \\ \hline
Requirement Priority & 1b \\ \hline
\multicolumn{2}{l}{Requirement Description and Discussion:} \\ \cdashline{1-2}
\end{tabular}

\textbf{Specification:} The Archive to Data Access Center communications
shall be highly available, with MTBF \textgreater{}
\textbf{archToDacNetMTBF}.

\begin{longtable}{p{4cm}p{12cm}}
\hline
Requirement Parameters & \textbf{archToDacNetMTBF = 180{{[}day{]}}} Mean Time Between Failures
for data service between Archive and DACs, averaged over a one-year
period. \\ \hline
Upper Level Requirement &
\begin{tabular}{cl}
DMS-REQ-0161 & Optimization of Cost, Reliability and Availability in Order \\
\end{tabular}
\\ \hline
\end{longtable}
}


\subsubsection{Test Cases Summary}
\begin{tabular}{p{3cm}p{2.5cm}p{2.5cm}p{3cm}p{4cm}}
\toprule
\href{https://jira.lsstcorp.org/secure/Tests.jspa\#/testCase/LVV-T201}{LVV-T201} & \multicolumn{4}{p{12cm}}{ Verify implementation of Archive to Data Access Center Network
Availability } \\ \hline
\textbf{Owner} & \textbf{Status} & \textbf{Version} & \textbf{Critical Event} & \textbf{Verification Type} \\ \hline
Jeff Kantor & Draft & 1 & false & Test \\ \hline
\end{tabular}
{\footnotesize
\textbf{Objective:}\\
Verify availability of archiving to Data Access Center Network using
test and historical data of or exceeding archToDacNetMTBF= 180{[}day{]}.
}

  
\newpage 
\subsection{[LVV-90] DMS-REQ-0190-V-01: Archive to Data Access Center Network Reliability }\label{lvv-90}

\begin{longtable}{cccc}
\hline
\textbf{Jira Link} & \textbf{Assignee} & \textbf{Status} & \textbf{Test Cases}\\ \hline
\href{https://jira.lsstcorp.org/browse/LVV-90}{LVV-90} &
Leanne Guy & Covered &
\begin{tabular}{c}
LVV-T202 \\
\end{tabular}
\\
\hline
\end{longtable}

\textbf{Verification Element Description:} \\
This requirement is verified by reconnecting and recovering transfer of
data between archive and DACs, after disconnecting fiber at an
intermediate location between archive and DACs.

{\footnotesize
\begin{tabular}{p{4cm}p{12cm}}
\hline
\multicolumn{2}{c}{\textbf{Requirement Details}}\\ \hline
Requirement ID & DMS-REQ-0190 \\ \hline
Requirement Priority & 1b \\ \hline
\multicolumn{2}{l}{Requirement Description and Discussion:} \\ \cdashline{1-2}
\end{tabular}

\textbf{Specification:} The Archive to Data Access Center communications
shall be highly reliable, with MTTR \textless{}
\textbf{archToDacNetMTTR}.

\begin{longtable}{p{4cm}p{12cm}}
\hline
Requirement Parameters & \textbf{archToDacNetMTTR = 48{{[}hour{]}}} Mean time to repair, measured
over a 1-yr period. \\ \hline
Upper Level Requirement &
\begin{tabular}{cl}
DMS-REQ-0161 & Optimization of Cost, Reliability and Availability in Order \\
\end{tabular}
\\ \hline
\end{longtable}
}


\subsubsection{Test Cases Summary}
\begin{tabular}{p{3cm}p{2.5cm}p{2.5cm}p{3cm}p{4cm}}
\toprule
\href{https://jira.lsstcorp.org/secure/Tests.jspa\#/testCase/LVV-T202}{LVV-T202} & \multicolumn{4}{p{12cm}}{ Verify implementation of Archive to Data Access Center Network
Reliability } \\ \hline
\textbf{Owner} & \textbf{Status} & \textbf{Version} & \textbf{Critical Event} & \textbf{Verification Type} \\ \hline
Jeff Kantor & Draft & 1 & false & Test \\ \hline
\end{tabular}
{\footnotesize
\textbf{Objective:}\\
Verify the reliability of Archive to Data Access Center Network by
demonstrating successful failover and capacity to the secondary part and
subsequent recovery to primary within or exceeding chToDacNetMTTR =
48{[}hour{]}.
}

  
\newpage 
\subsection{[LVV-91] DMS-REQ-0191-V-01: Archive to Data Access Center Network Secondary Link }\label{lvv-91}

\begin{longtable}{cccc}
\hline
\textbf{Jira Link} & \textbf{Assignee} & \textbf{Status} & \textbf{Test Cases}\\ \hline
\href{https://jira.lsstcorp.org/browse/LVV-91}{LVV-91} &
Leanne Guy & Covered &
\begin{tabular}{c}
LVV-T203 \\
\end{tabular}
\\
\hline
\end{longtable}

\textbf{Verification Element Description:} \\
This requirement is verified by reconnecting and recovering transfer of
data between archive and DACs, after disconnecting fiber at an
intermediate location between archive and DACs.

{\footnotesize
\begin{tabular}{p{4cm}p{12cm}}
\hline
\multicolumn{2}{c}{\textbf{Requirement Details}}\\ \hline
Requirement ID & DMS-REQ-0191 \\ \hline
Requirement Priority & 1b \\ \hline
\multicolumn{2}{l}{Requirement Description and Discussion:} \\ \cdashline{1-2}
\end{tabular}

\textbf{Specification:} The Archive to Data Access Center communications
shall provide secondary link or transport mechanism (e.g. protected
circuit) for operations support and ``catch up'' in the event of
extended outage.

\begin{longtable}{p{4cm}p{12cm}}
\hline
Upper Level Requirement &
\begin{tabular}{cl}
DMS-REQ-0189 & Archive to Data Access Center Network Availability \\
DMS-REQ-0190 & Archive to Data Access Center Network Reliability \\
\end{tabular}
\\ \hline
\end{longtable}
}


\subsubsection{Test Cases Summary}
\begin{tabular}{p{3cm}p{2.5cm}p{2.5cm}p{3cm}p{4cm}}
\toprule
\href{https://jira.lsstcorp.org/secure/Tests.jspa\#/testCase/LVV-T203}{LVV-T203} & \multicolumn{4}{p{12cm}}{ Verify implementation of Archive to Data Access Center Network Secondary
Link } \\ \hline
\textbf{Owner} & \textbf{Status} & \textbf{Version} & \textbf{Critical Event} & \textbf{Verification Type} \\ \hline
Kian-Tat Lim & Draft & 1 & false & Test \\ \hline
\end{tabular}
{\footnotesize
\textbf{Objective:}\\
Verify the Archive to Data Access Center Network via Secondary Link by
simulating a failure on the primary path and capacity on the secondary
path.
}

  
\newpage 
\subsection{[LVV-183] DMS-REQ-0352-V-01: Base Wireless LAN (WiFi) }\label{lvv-183}

\begin{longtable}{cccc}
\hline
\textbf{Jira Link} & \textbf{Assignee} & \textbf{Status} & \textbf{Test Cases}\\ \hline
\href{https://jira.lsstcorp.org/browse/LVV-183}{LVV-183} &
Leanne Guy & Covered &
\begin{tabular}{c}
LVV-T192 \\
\end{tabular}
\\
\hline
\end{longtable}

\textbf{Verification Element Description:} \\
At Base Facility, connect to WiFi, test connection speed, i.e. send
email, browse web, and retrieve files from the Internet.

{\footnotesize
\begin{tabular}{p{4cm}p{12cm}}
\hline
\multicolumn{2}{c}{\textbf{Requirement Details}}\\ \hline
Requirement ID & DMS-REQ-0352 \\ \hline
Requirement Priority & 2 \\ \hline
\multicolumn{2}{l}{Requirement Description and Discussion:} \\ \cdashline{1-2}
\end{tabular}

\textbf{Specification:} The Base LAN shall provide \textbf{minBaseWiFi}
Wireless LAN (WiFi) and Wireless Access Points in the Base Facility to
support connectivity of individual user's computers to the network
backbones.

\begin{longtable}{p{4cm}p{12cm}}
\hline
Requirement Parameters & \textbf{minBaseWifi = 1000{{[}megabit per second{]}}} Maximum allowable
outage of active DM production. \\ \hline
Upper Level Requirement &
\begin{tabular}{cl}
OSS-REQ-0003 & The Base Facility \\
\end{tabular}
\\ \hline
\end{longtable}
}


\subsubsection{Test Cases Summary}
\begin{tabular}{p{3cm}p{2.5cm}p{2.5cm}p{3cm}p{4cm}}
\toprule
\href{https://jira.lsstcorp.org/secure/Tests.jspa\#/testCase/LVV-T192}{LVV-T192} & \multicolumn{4}{p{12cm}}{ Verify implementation of Base Wireless LAN (WiFi) } \\ \hline
\textbf{Owner} & \textbf{Status} & \textbf{Version} & \textbf{Critical Event} & \textbf{Verification Type} \\ \hline
Jeff Kantor & Draft & 1 & false & Test \\ \hline
\end{tabular}
{\footnotesize
\textbf{Objective:}\\
Verify as-built wireless network at the Base Facility supports
minBaseWiFi bandwidth (1000 Mbs).
}

  
\newpage 
\subsection{[LVV-18491] DMS-REQ-0352-V-02: Base Voice Over IP (VOIP) }\label{lvv-18491}

\begin{longtable}{cccc}
\hline
\textbf{Jira Link} & \textbf{Assignee} & \textbf{Status} & \textbf{Test Cases}\\ \hline
\href{https://jira.lsstcorp.org/browse/LVV-18491}{LVV-18491} &
Leanne Guy & Covered &
\begin{tabular}{c}
LVV-T181 \\
\end{tabular}
\\
\hline
\end{longtable}

\textbf{Verification Element Description:} \\
Verify (a) plannned and (b) as-built VOIP at the Base Facility is
operational and performs as expected (i.e. sufficient number of
extensions allocated properly, no frequent drop-outs, no frequent
jaggies on video, etc.). Test voice calls and videoconferening.

{\footnotesize
\begin{tabular}{p{4cm}p{12cm}}
\hline
\multicolumn{2}{c}{\textbf{Requirement Details}}\\ \hline
Requirement ID & DMS-REQ-0352 \\ \hline
Requirement Priority & 2 \\ \hline
\multicolumn{2}{l}{Requirement Description and Discussion:} \\ \cdashline{1-2}
\end{tabular}

\textbf{Specification:} The Base LAN shall provide \textbf{minBaseWiFi}
Wireless LAN (WiFi) and Wireless Access Points in the Base Facility to
support connectivity of individual user's computers to the network
backbones.

\begin{longtable}{p{4cm}p{12cm}}
\hline
Requirement Parameters & \textbf{minBaseWifi = 1000{{[}megabit per second{]}}} Maximum allowable
outage of active DM production. \\ \hline
Upper Level Requirement &
\begin{tabular}{cl}
OSS-REQ-0003 & The Base Facility \\
\end{tabular}
\\ \hline
\end{longtable}
}


\subsubsection{Test Cases Summary}
\begin{tabular}{p{3cm}p{2.5cm}p{2.5cm}p{3cm}p{4cm}}
\toprule
\href{https://jira.lsstcorp.org/secure/Tests.jspa\#/testCase/LVV-T181}{LVV-T181} & \multicolumn{4}{p{12cm}}{ Verify Base Voice Over IP (VOIP) } \\ \hline
\textbf{Owner} & \textbf{Status} & \textbf{Version} & \textbf{Critical Event} & \textbf{Verification Type} \\ \hline
Jeff Kantor & Draft & 1 & false & Test \\ \hline
\end{tabular}
{\footnotesize
\textbf{Objective:}\\
Verify as-built VOIP at the Base Facility is operational and performs as
expected (i.e. sufficient number of extensions allocated properly, no
frequent drop-outs, no frequent jaggies on video, etc.) on both voice
calls and videoconferening.
}

  

\newpage
\appendix
\section{Traceability}
\label{sec:trace}

\begin{longtable}{ccc}
\hline
\textbf{Requirements} & \textbf{Verification Elements} & \textbf{Test Cases} \\ \hline
 DMS-REQ-0168  &
 LVV-71  &
LVV-T1097 \\
 &
 &
LVV-T2338 \\
\hline
 DMS-REQ-0171  &
 LVV-73  &
LVV-T1168 \\
 &
 &
LVV-T1612 \\
\hline
 DMS-REQ-0172  &
 LVV-74  &
LVV-T185 \\
\hline
 DMS-REQ-0173  &
 LVV-75  &
LVV-T186 \\
\hline
 DMS-REQ-0174  &
 LVV-76  &
LVV-T187 \\
\hline
 DMS-REQ-0175  &
 LVV-77  &
LVV-T188 \\
\hline
 DMS-REQ-0180  &
 LVV-81  &
LVV-T193 \\
\hline
 DMS-REQ-0181  &
 LVV-82  &
LVV-T194 \\
\hline
 DMS-REQ-0182  &
 LVV-83  &
LVV-T195 \\
\hline
 DMS-REQ-0183  &
 LVV-84  &
LVV-T196 \\
\hline
 DMS-REQ-0188  &
 LVV-88  &
LVV-T200 \\
\hline
 DMS-REQ-0189  &
 LVV-89  &
LVV-T201 \\
\hline
 DMS-REQ-0190  &
 LVV-90  &
LVV-T202 \\
\hline
 DMS-REQ-0191  &
 LVV-91  &
LVV-T203 \\
\hline
 DMS-REQ-0352  &
 LVV-183  &
LVV-T192 \\
 \cdashline{2-3}  &
 LVV-18491  &
LVV-T181 \\
\hline
\end{longtable}

Note that some of the requirements listed in this traceability table may be related with additional
Verification Elements not in the scope of
\textit{ DM } component
\textit{ NETWORK } subcomponent
Verification,
and therefore not listed here.


\newpage
\section{References\label{sect:references}}
\renewcommand{\refname}{}
\bibliography{lsst,refs,books,refs_ads,local.bib}

\newpage
\section{Acronyms \label{sect:acronyms}} % include acronyms.tex generated by the generateAcronyms.py (in texmf/scripts)
\addtocounter{table}{-1}
\begin{longtable}{p{0.145\textwidth}p{0.8\textwidth}}\hline
\textbf{Acronym} & \textbf{Description}  \\\hline

BDC &  Base Data Center \\\hline
BERT & Bit Error Rate Tester \\\hline
CCS & Camera Control System \\\hline
CHI & Chicago \\\hline
CHMPGN & Champaign (Illinois) \\\hline
CISS & Computer Infrastructure Services South (part of the former NOAO Cerro Tololo Inter-american Observatory (CTIO), now merged into NSF’S OIR Lab Central Operating Services) \\\hline
CTIO & Cerro Tololo Inter-American Observatory \\\hline
DAC & Data Access Center \\\hline
DAQ & Data Acquisition System \\\hline
DM & Data Management \\\hline
DMCS & Data Management Control System \\\hline
DMS & Data Management Subsystem \\\hline
DMS-REQ & Data Management top level requirements (\citeds{LSE-61}) \\\hline
DMSR & DM System Requirements; LSE-61 \\\hline
DMSSIT & DM Subsystem Integration Test \\\hline
DMTR & DM Test (Plan and) Report \\\hline
DTN & Data Transfer Node \\\hline
DWDM & Dense Wave Division Multiplex \\\hline
EFD & Engineering and Facility Database \\\hline
EPO & Education and Public Outreach \\\hline
FIU & Florida International University \\\hline
FL & Florida \\\hline
HL & Higher Level \\\hline
IP & Internet Protocol \\\hline
ISO & International Standards Organization \\\hline
IT & Information Technology \\\hline
LAN & Local Area Network \\\hline
LATISS & LSST Atmospheric Transmission Imager and Slitless Spectrograph \\\hline
LDF & LSST Data Facility \\\hline
LDM & LSST Data Management (Document Handle) \\\hline
LHN & Long-Haul Networks \\\hline
LL & Lower Level \\\hline
LS & La Serena \\\hline
LSE & LSST Systems Engineering (Document Handle) \\\hline
LSST & Large Synoptic Survey Telescope \\\hline
LVV & LSST Verification and Validation (Jira project) \\\hline
MTBF & Mean Time Between Failures \\\hline
MTTR & Mean Time to Repair \\\hline
NCSA & National Center for Supercomputing Applications \\\hline
NET & Network Engineering Team \\\hline
OCS & Observatory Control System \\\hline
OSI & Open System Interconnect \\\hline
OSS & Observatory System Specifications; LSE-30 \\\hline
OTDR & Optical Time Domain Reflectometer \\\hline
PMCS & Project Management Controls System \\\hline
REUNA & Red Universitaria Nacional \\\hline
SC & Science Collaboration \\\hline
SCL & Santiago, Chile \\\hline
SIT & LSST System Integration Test \\\hline
SL & Same Level \\\hline
SLAC & SLAC National Accelerator Lab \\\hline
TCS & Telescope Control System \\\hline
US & United States \\\hline
VNVD & Vera C Rubin Observatory Network Verification Document \\\hline
VOIP & Voice Over Internet Protocol \\\hline
WBS & Work Breakdown Structure \\\hline
\end{longtable}


%\section{Glossary \label{sect:glossary}}
%\renewcommand{\refname}{}
%\printglossaries

\end{document}
