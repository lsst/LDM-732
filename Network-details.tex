%%%%%%%%%%%%%%%%%%%%%%%%%%%%%%%%%%%%%%%%%%%%%%%%%%%%%%%%%%%%%%%%%%%%%%%%%%%%%%%%%%%%%%%%%%%%%%%
% do not edit
%%%%%%%%%%%%%%%%%%%%%%%%%%%%%%%%%%%%%%%%%%%%%%%%%%%%%%%%%%%%%%%%%%%%%%%%%%%%%%%%%%%%%%%%%%%%%%%

\documentclass[DM]{lsstdoc}
\usepackage{enumitem}
\usepackage{booktabs}
\usepackage{longtable}
\usepackage{arydshln}
\usepackage{afterpage}
\usepackage{pdflscape}
\usepackage{graphicx}
\usepackage{lipsum}

\input{meta.tex}

\author{}

\DeclareRobustCommand{\cmdkey}{\raisebox{-.035em}{\includegraphics[height=.75em]{figures/cmdkey}}}

\begin{document}

\date{\today}

\providecommand{\tightlist}{
  \setlength{\itemsep}{0pt}\setlength{\parskip}{0pt}}

\def\product{LSST Data Management}


\title{Verification Elelements Details for the Network Component in DM Subsystem}
\mkshorttitle

This is a detailed overview of the Verification Elements and relevant associated information,
the Network component in DM subsystem.
It is provided for convenience as a working document.
The information presented here is officially baselined in {\lsstDocType}-{\lsstDocNum} --
the Verification Elements baseline document -- available at
\url{https://\lsstDocType-\lsstDocNum.lsst.io}.
Test case information is baselined in the  {\testspec} test specification, available at
\url{https://\testspec.lsst.io}.
Official releases of both documents are also available in Docushare.
Please always use {\lsstDocType}-{\lsstDocNum} and {\testspec} official releases for reference.

This report is updated together with the the verification elements baseline document, {\lsstDocType}-{\lsstDocNum}.
Therefore, verification elements information will be always up-to-date.
Test cases information instead may be outdated, since test cases may be subject to changes during future phases
of the V\&V activities.


\section{Summary Overview}

\begin{longtable}{cccc}
\hline
\textbf{Requirements} & & \textbf{Verification Elements} & \textbf{Test Cases} \\ \hline
 DMS-REQ-0168 &
  {\scriptsize \ref{lvv-71} } & LVV-71({\scriptsize Gregory Dubois-Felsmann })
 &
LVV-T1097 ({\scriptsize Jeff Kantor }) \\
\hline
 DMS-REQ-0171 &
  {\scriptsize \ref{lvv-73} } & LVV-73({\scriptsize Robert Gruendl })
 &
LVV-T1168 ({\scriptsize Jeff Kantor }) \\
 &
  &
 &
LVV-T1612 ({\scriptsize Jeff Kantor }) \\
\hline
 DMS-REQ-0172 &
  {\scriptsize \ref{lvv-74} } & LVV-74({\scriptsize Robert Gruendl })
 &
LVV-T185 ({\scriptsize Jeff Kantor }) \\
\hline
 DMS-REQ-0173 &
  {\scriptsize \ref{lvv-75} } & LVV-75({\scriptsize Robert Gruendl })
 &
LVV-T186 ({\scriptsize Jeff Kantor }) \\
\hline
 DMS-REQ-0174 &
  {\scriptsize \ref{lvv-76} } & LVV-76({\scriptsize Robert Gruendl })
 &
LVV-T187 ({\scriptsize Jeff Kantor }) \\
\hline
 DMS-REQ-0175 &
  {\scriptsize \ref{lvv-77} } & LVV-77({\scriptsize Robert Gruendl })
 &
LVV-T188 ({\scriptsize Jeff Kantor }) \\
\hline
 DMS-REQ-0180 &
  {\scriptsize \ref{lvv-81} } & LVV-81({\scriptsize Robert Gruendl })
 &
LVV-T193 ({\scriptsize Jeff Kantor }) \\
\hline
 DMS-REQ-0181 &
  {\scriptsize \ref{lvv-82} } & LVV-82({\scriptsize Robert Gruendl })
 &
LVV-T194 ({\scriptsize Jeff Kantor }) \\
\hline
 DMS-REQ-0182 &
  {\scriptsize \ref{lvv-83} } & LVV-83({\scriptsize Robert Gruendl })
 &
LVV-T195 ({\scriptsize Jeff Kantor }) \\
\hline
 DMS-REQ-0183 &
  {\scriptsize \ref{lvv-84} } & LVV-84({\scriptsize Robert Gruendl })
 &
LVV-T196 ({\scriptsize Jeff Kantor }) \\
\hline
 DMS-REQ-0188 &
  {\scriptsize \ref{lvv-88} } & LVV-88({\scriptsize Robert Gruendl })
 &
LVV-T200 ({\scriptsize Jeff Kantor }) \\
\hline
 DMS-REQ-0189 &
  {\scriptsize \ref{lvv-89} } & LVV-89({\scriptsize Robert Gruendl })
 &
LVV-T201 ({\scriptsize Jeff Kantor }) \\
\hline
 DMS-REQ-0190 &
  {\scriptsize \ref{lvv-90} } & LVV-90({\scriptsize Robert Gruendl })
 &
LVV-T202 ({\scriptsize Jeff Kantor }) \\
\hline
 DMS-REQ-0191 &
  {\scriptsize \ref{lvv-91} } & LVV-91({\scriptsize Robert Gruendl })
 &
LVV-T203 ({\scriptsize Kian-Tat Lim }) \\
\hline
 DMS-REQ-0352 &
  {\scriptsize \ref{lvv-183} } & LVV-183({\scriptsize Robert Gruendl })
 &
LVV-T192 ({\scriptsize Jeff Kantor }) \\
 \cdashline{2-4}  &
  {\scriptsize \ref{lvv-18491} } & LVV-18491({\scriptsize Robert Gruendl })
 &
LVV-T181 ({\scriptsize Jeff Kantor }) \\
\hline
\end{longtable}

\section{Verification Elements Details}
\label{sec:ves}

The following is the list of verification elements defined in the context of the Network component of the DM subsystem.

\subsection{[LVV-71] DMS-REQ-0168-V-01: Summit Facility Data Communications }\label{lvv-71}

\begin{longtable}{ccccc}
\hline
\textbf{Jira Link} & \textbf{Assignee} & \textbf{Status} & \textbf{Priority} & \textbf{Test Cases}\\ \hline
\href{https://jira.lsstcorp.org/browse/LVV-71}{LVV-71} &
Gregory Dubois-Felsmann & Not Covered & 1a &
\begin{tabular}{c}
LVV-T1097 \\
\end{tabular}
\\
\hline
\end{longtable}

\textbf{Verification Element Description:} \\
Verify that:

\begin{itemize}
\tightlist
\item
  Summit - Base Network has been properly implemented in Summit and Base
  facilities
\item
  Summit - Base Network is properly integrated with Summit Control
  Network and DAQ/Camera Data Backbone
\end{itemize}

Verify that OCS/DMCS triggers read-out from DAQ and queries EFD. verify
that data from EFD and camera are accepted and transferred to the Summit
DWDM. Requirement does not include data transfer to base
(\href{https://jira.lsstcorp.org/browse/LVV-73}{LVV-73}) or from base to
archive center (\href{https://jira.lsstcorp.org/browse/LVV-81}{LVV-81},
\href{https://jira.lsstcorp.org/browse/LVV-82}{LVV-82},
\href{https://jira.lsstcorp.org/browse/LVV-83}{LVV-83}).

{\footnotesize
\begin{longtable}{p{3cm}p{13cm}}
\hline
\multicolumn{2}{c}{\textbf{Upstream Requirements}}\\ \hline
Requirement ID & DMS-REQ-0168 \\ \cdashline{1-2}
Requirement Description & \textbf{Specification:} The DMS shall provide data communications
infrastructure to accept science data and associated metadata read-outs,
and the collection of ancillary and engineering data, for transfer to
the base facility. \\ \cdashline{1-2}
Requirement Priority & 1a \\ \cdashline{1-2}
Upper Level Requirement &
\begin{tabular}{cl}
OSS-REQ-0002 & The Summit Facility \\
\end{tabular}
\\ \hline
\end{longtable}
}

\subsubsection{[LVV-T1097] Verify Summit Facility Network Implementation }

\begin{longtable}{cccccc}
\hline
\multicolumn{6}{c}{\textbf{Test Case Summary}} \\ \hline
\textbf{Jira Link} & \textbf{Owner} & \textbf{Status} & \textbf{Version} & \textbf{Critical Event} &
\textbf{Verification Type} \\ \hline
\href{https://jira.lsstcorp.org/secure/Tests.jspa\#/testCase/LVV-T1097}{LVV-T1097} &
Jeff Kantor & Draft & 1 & false & Test
\\ \hline
\end{longtable}

\textbf{Objective:} \\
Verify that data acquired by a AuxTel DAQ can be transferred to Summit
DWDM and loaded in the EFD without problems.

\textbf{Precondition:} \\
\begin{enumerate}
\tightlist
\item
  Summit Control Network and Camera Data Backbone installed and
  operating properly.
\item
  Summit - Base Network installed and operating properly.
\item
  AuxTel hardware and control systems are functional with LATISS. AuxTel
  TCS, AuxTel EFD, AuxTel CCS, AuxTel DAQ are connected via Control
  Network on Summit to Rubin Observatory DWDM (with at least 2 x 10 Gbps
  ethernet port client cards).
\item
  AuxTel Archiver/forwarders installed in Summit and operating properly.
\item
  As-built documentation for all of the above is available.
\end{enumerate}

\textbf{Predecessors:} \\
PMCS DMTC-7400-2400 Complete\\
PMCS T\&SC-2600-1545 Complete

\textbf{Test Personnel:} \\
Ron Lambert (Rubin Observatory), Kian-Tat Lim (Rubin Observatory), Matt
Kollross (NCSA), Tony Johnson (SLAC), Gregg Thayer (SLAC)

\textbf{Test Procedure}
    \begin{longtable}[]{p{1.3cm}p{2cm}p{13cm}}
    %\toprule
    Step & \multicolumn{2}{@{}l}{Description, Input Data and Expected Result} \\ \toprule
    \endhead
            \multirow{3}{*}{\parbox{1.3cm}{ 1
} }
& {\small Description} &
\begin{minipage}[t]{13cm}{\scriptsize
Verify the pre-conditions have been satisfied

\vspace{\dp0}
} \end{minipage} \\ \cdashline{2-3}
& {\small Test Data} &
\begin{minipage}[t]{13cm}{\scriptsize
NA

\vspace{\dp0}
} \end{minipage}
\\ \cdashline{2-3}
& {\small Expected Result} &
\begin{minipage}[t]{13cm}{\scriptsize
Pre-conditions are satisfied.

\vspace{\dp0}
} \end{minipage}
\\ \hdashline
        \\ \midrule
            \multirow{3}{*}{\parbox{1.3cm}{ 2
} }
& {\small Description} &
\begin{minipage}[t]{13cm}{\scriptsize
Control the AuxTel through a night of Observing. ~While observing, read
out LATISS data and transfer to Rubin Observatory Summit DWDM while
monitoring latency.

\vspace{\dp0}
} \end{minipage} \\ \cdashline{2-3}
& {\small Test Data} &
\begin{minipage}[t]{13cm}{\scriptsize
LATISS images and metadata

\vspace{\dp0}
} \end{minipage}
\\ \cdashline{2-3}
& {\small Expected Result} &
\begin{minipage}[t]{13cm}{\scriptsize
Data is fed to DWDM without delays or errors.

\vspace{\dp0}
} \end{minipage}
\\ \hdashline
        \\ \midrule
            \multirow{3}{*}{\parbox{1.3cm}{ 3
} }
& {\small Description} &
\begin{minipage}[t]{13cm}{\scriptsize
Verify that data acquired by a AuxTel DAQ can be transferred ~and loaded
in EFD without problems.

\vspace{\dp0}
} \end{minipage} \\ \cdashline{2-3}
& {\small Test Data} &
\begin{minipage}[t]{13cm}{\scriptsize
LATISS images and metadata

\vspace{\dp0}
} \end{minipage}
\\ \cdashline{2-3}
& {\small Expected Result} &
\begin{minipage}[t]{13cm}{\scriptsize
Examine the EFD to ensure that the data has been loaded properly.

\vspace{\dp0}
} \end{minipage}
\\ \hdashline
        \\ \midrule
    \end{longtable}


\subsection{[LVV-73] DMS-REQ-0171-V-01: Summit to Base Network }\label{lvv-73}

\begin{longtable}{ccccc}
\hline
\textbf{Jira Link} & \textbf{Assignee} & \textbf{Status} & \textbf{Priority} & \textbf{Test Cases}\\ \hline
\href{https://jira.lsstcorp.org/browse/LVV-73}{LVV-73} &
Robert Gruendl & Not Covered & 1a &
\begin{tabular}{c}
LVV-T1168 \\
LVV-T1612 \\
\end{tabular}
\\
\hline
\end{longtable}

\textbf{Verification Element Description:} \\
This requirement must be tested in sequence and collect performance
metrics (both DAQ and Control sides unless noted):

\begin{enumerate}
\tightlist
\item
  ISO OSI Layer 1 Physical (fibers with test data from OTDR, AURA does
  test)
\item
  ISO OSI Layer 2 Data Link (DWDM equipment, line cards, with test data
  from multi-channel/lightwave/channel analyzer, Installer does test,
  AURA certify)
\item
  ISO Layer 3 minimal (DWDM with 2 x 10 Gbps ethernet port client cards
  with test data from 4 windows test boxes, 2 on each side, Installer
  does test, AURA certify, can repeat as part of \#4 with DAQ)
\item
  ISO Layer 3 full (22 x 10 Gbps ethernet ports on DAQ side with test
  data from DAQ test stand, AURA, Camera DAQ team do test). Transfer
  data between summit and base over uninterrupted 1 day period.
  Â~Demonstrate transfer of data at or exceeding rates specified in
  \citeds{LDM-142}.
\end{enumerate}

{\footnotesize
\begin{longtable}{p{3cm}p{13cm}}
\hline
\multicolumn{2}{c}{\textbf{Upstream Requirements}}\\ \hline
Requirement ID & DMS-REQ-0171 \\ \cdashline{1-2}
Requirement Description & \textbf{Specification:} The DMS shall provide communications
infrastructure between the Summit Facility and the Base Facility
sufficient to carry scientific data and associated metadata for each
image in no more than time \textbf{summToBaseMaxTransferTime}. \\ \cdashline{1-2}
Requirement Parameters & \textbf{summToBaseMaxTransferTime = 2{{[}second{]}}} Maximum time
interval to transfer a full Crosstalk Corrected Exposure and all related
metadata from the Summit Facility to the Base facility. \\ \cdashline{1-2}
Requirement Priority & 1a \\ \cdashline{1-2}
Upper Level Requirement &
\begin{tabular}{cl}
OSS-REQ-0003 & The Base Facility \\
OSS-REQ-0127 & Level 1 Data Product Availability \\
\end{tabular}
\\ \hline
\end{longtable}
}

\subsubsection{[LVV-T1168] Verify Summit - Base Network Integration }

\begin{longtable}{cccccc}
\hline
\multicolumn{6}{c}{\textbf{Test Case Summary}} \\ \hline
\textbf{Jira Link} & \textbf{Owner} & \textbf{Status} & \textbf{Version} & \textbf{Critical Event} &
\textbf{Verification Type} \\ \hline
\href{https://jira.lsstcorp.org/secure/Tests.jspa\#/testCase/LVV-T1168}{LVV-T1168} &
Jeff Kantor & Approved & 1 & false & Inspection
\\ \hline
\end{longtable}

\textbf{Objective:} \\
3 phases done (in collaboration with equipment/installation vendors):

\begin{enumerate}
\tightlist
\item
  Installation of fiber optic cables and Optical Time Domain Reflector
  (OTDR) fiber testing (completed 20170602
  \href{https://docushare.lsstcorp.org/docushare/dsweb/Get/Document-26270/RD10\%20Report\%20of\%20delivery\%20of\%20LS\%20-\%20AG\%20fiber\%20from\%20Telefonica\%20to\%20REUNA.pdf}{REUNA
  deliverable RD10})
\item
  Installation of AURA DWDM and Data Transfer Node (DTN) (completed
  20171218
  \href{https://docushare.lsst.org/docushare/dsweb/Get/DMTR-82/DMTR-82.pdf}{DMTR-82})
\item
  Installation of LSST DWDM and Bit Error Rate Tester (BERT) data
  (completed 20190505
  \href{https://docushare.lsstcorp.org/docushare/dsweb/View/Collection-7743}{collection-7743},
  20191108
  \href{https://docushare.lsstcorp.org/docushare/dsweb/Get/Document-35302/DAQ\%20DWDM\%20connection\%20tests\%2020191109.pptx}{DAQ
  DWDM Connection Tests})
\end{enumerate}

\textbf{Precondition:} \\
PMCS DMTC-7400-2330 COMPLETE\\
By phase:

\begin{enumerate}
\tightlist
\item
  Posts from Cerro Pachon to AURA Gatehouse repaired/improved. ~Fiber
  installed on posts from Cerro Pachon to AURA Gatehouse. ~Fiber
  installed from AURA Gatehouse to AURA compound in La Serena. OTDR
  purchased.
\item
  AURA DWDM installed in caseta on Cerro Pachon and in existing computer
  room in La Serena. ~DTN installed in La Serena. ~DTN loaded with
  software and test data staged.
\item
  Base Data Center (BDC) ready for installation of LSST DWDM. ~Fiber
  connecting existing computer room to BDC. ~LSST DWDM equipment
  installed in Summit Computer Room and BDC.
\end{enumerate}

\textbf{Predecessors:} \\
See pre-conditions by phase above.

\textbf{Test Personnel:} \\
Ron Lambert (LSST), Albert Astudillo (REUNA), Mauricio Rojas
(CTIO/CISS), Raylex, Coriant, Telefonica contractors

\textbf{Test Procedure}
    \begin{longtable}[]{p{1.3cm}p{2cm}p{13cm}}
    %\toprule
    Step & \multicolumn{2}{@{}l}{Description, Input Data and Expected Result} \\ \toprule
    \endhead
            \multirow{3}{*}{\parbox{1.3cm}{ 1
} }
& {\small Description} &
\begin{minipage}[t]{13cm}{\scriptsize
Test optical fiber with OTDR

\vspace{\dp0}
} \end{minipage} \\ \cdashline{2-3}
& {\small Test Data} &
\begin{minipage}[t]{13cm}{\scriptsize
OTDR generated optical data

\vspace{\dp0}
} \end{minipage}
\\ \cdashline{2-3}
& {\small Expected Result} &
\begin{minipage}[t]{13cm}{\scriptsize
Fiber tested to within acceptable Db.

\vspace{\dp0}
} \end{minipage}
\\ \hdashline
        \\ \midrule
            \multirow{3}{*}{\parbox{1.3cm}{ 2
} }
& {\small Description} &
\begin{minipage}[t]{13cm}{\scriptsize
Test AURA DWDM

\vspace{\dp0}
} \end{minipage} \\ \cdashline{2-3}
& {\small Test Data} &
\begin{minipage}[t]{13cm}{\scriptsize
DTN perfSonar generated data

\vspace{\dp0}
} \end{minipage}
\\ \cdashline{2-3}
& {\small Expected Result} &
\begin{minipage}[t]{13cm}{\scriptsize
Summit - Base bandwidth and latency within specifications

\vspace{\dp0}
} \end{minipage}
\\ \hdashline
        \\ \midrule
            \multirow{3}{*}{\parbox{1.3cm}{ 3
} }
& {\small Description} &
\begin{minipage}[t]{13cm}{\scriptsize
Test LSST DWDM

\vspace{\dp0}
} \end{minipage} \\ \cdashline{2-3}
& {\small Test Data} &
\begin{minipage}[t]{13cm}{\scriptsize
BERT generated data

\vspace{\dp0}
} \end{minipage}
\\ \cdashline{2-3}
& {\small Expected Result} &
\begin{minipage}[t]{13cm}{\scriptsize
Summit - Base bandwidth, latency, bit error rate within specifications

\vspace{\dp0}
} \end{minipage}
\\ \hdashline
        \\ \midrule
    \end{longtable}

\subsubsection{[LVV-T1612] Verify Summit - Base Network Integration (System Level) }

\begin{longtable}{cccccc}
\hline
\multicolumn{6}{c}{\textbf{Test Case Summary}} \\ \hline
\textbf{Jira Link} & \textbf{Owner} & \textbf{Status} & \textbf{Version} & \textbf{Critical Event} &
\textbf{Verification Type} \\ \hline
\href{https://jira.lsstcorp.org/secure/Tests.jspa\#/testCase/LVV-T1612}{LVV-T1612} &
Jeff Kantor & Draft & 1 & false & Inspection
\\ \hline
\end{longtable}

\textbf{Objective:} \\
Verify ISO Layer 3 full (22 x 10 Gbps ethernet ports on DAQ side with
test data from DAQ test stand, AURA, Camera DAQ team do test).
Demonstrate transfer of data at or exceeding rates specified in \citeds{LDM-142}.

\textbf{Precondition:} \\
PMCS DMTC-7400-2400 COMPLETE\\
\href{https://jira.lsstcorp.org/secure/Tests.jspa\#/testCase/1401}{LVV-T1168}
Passed\\
Full Camera DAQ installed on summit and loaded with data.\\
Archiver/forwarders installed at Base.\\
As-built documentation for all of the above is
available.\\[2\baselineskip]

\textbf{Predecessors:} \\
See pre-conditions.

\textbf{Test Personnel:} \\
Ron Lambert (LSST), Greg Thayer (SLAC)

\textbf{Test Procedure}
    \begin{longtable}[]{p{1.3cm}p{2cm}p{13cm}}
    %\toprule
    Step & \multicolumn{2}{@{}l}{Description, Input Data and Expected Result} \\ \toprule
    \endhead
            \multirow{3}{*}{\parbox{1.3cm}{ 1
} }
& {\small Description} &
\begin{minipage}[t]{13cm}{\scriptsize
Verify Pre-conditions are satisfied.

\vspace{\dp0}
} \end{minipage} \\ \cdashline{2-3}
& {\small Test Data} &
\begin{minipage}[t]{13cm}{\scriptsize
NA

\vspace{\dp0}
} \end{minipage}
\\ \cdashline{2-3}
& {\small Expected Result} &
\begin{minipage}[t]{13cm}{\scriptsize
Pre-conditions are satisfied.

\vspace{\dp0}
} \end{minipage}
\\ \hdashline
        \\ \midrule
            \multirow{3}{*}{\parbox{1.3cm}{ 2
} }
& {\small Description} &
\begin{minipage}[t]{13cm}{\scriptsize
Transfer data between summit and base over uninterrupted 1 day period.
~Monitor transfer of data at or exceeding rates specified in LDM-142.

\vspace{\dp0}
} \end{minipage} \\ \cdashline{2-3}
& {\small Test Data} &
\begin{minipage}[t]{13cm}{\scriptsize
DAQ pre-loaded data

\vspace{\dp0}
} \end{minipage}
\\ \cdashline{2-3}
& {\small Expected Result} &
\begin{minipage}[t]{13cm}{\scriptsize
Data transfers at or exceeding rates specified in LDM-142.

\vspace{\dp0}
} \end{minipage}
\\ \hdashline
        \\ \midrule
    \end{longtable}


\subsection{[LVV-74] DMS-REQ-0172-V-01: Summit to Base Network Availability }\label{lvv-74}

\begin{longtable}{ccccc}
\hline
\textbf{Jira Link} & \textbf{Assignee} & \textbf{Status} & \textbf{Priority} & \textbf{Test Cases}\\ \hline
\href{https://jira.lsstcorp.org/browse/LVV-74}{LVV-74} &
Robert Gruendl & Not Covered & 1a &
\begin{tabular}{c}
LVV-T185 \\
\end{tabular}
\\
\hline
\end{longtable}

\textbf{Verification Element Description:} \\
This requirement needs the network link to be active for a calculated
amount of time (at least 1 week) without failure. Will require
extrapolating from test and historical data as failures are rare.
Monthly operating statistics will be acquired during commissioning.
Demonstrate transfer of data at or exceeding rates specified in \citeds{LDM-142},
verify achieved average and peak throughput and latency.

{\footnotesize
\begin{longtable}{p{3cm}p{13cm}}
\hline
\multicolumn{2}{c}{\textbf{Upstream Requirements}}\\ \hline
Requirement ID & DMS-REQ-0172 \\ \cdashline{1-2}
Requirement Description & \textbf{Specification:} The Summit to Base communications shall be
highly available, with Mean Time Between Failures (MTBF) \textgreater{}
\textbf{summToBaseNetMTBF}. \\ \cdashline{1-2}
Requirement Parameters & \textbf{summToBaseNetMTBF = 90{{[}day{]}}} Mean time between failures,
measured over a 1-yr period. \\ \cdashline{1-2}
Requirement Priority & 1b \\ \cdashline{1-2}
Upper Level Requirement &
\begin{tabular}{cl}
OSS-REQ-0373 & Unscheduled Downtime Subsystem Allocations \\
DMS-REQ-0161 & Optimization of Cost, Reliability and Availability in Order \\
\end{tabular}
\\ \hline
\end{longtable}
}

\subsubsection{[LVV-T185] Verify implementation of Summit to Base Network Availability }

\begin{longtable}{cccccc}
\hline
\multicolumn{6}{c}{\textbf{Test Case Summary}} \\ \hline
\textbf{Jira Link} & \textbf{Owner} & \textbf{Status} & \textbf{Version} & \textbf{Critical Event} &
\textbf{Verification Type} \\ \hline
\href{https://jira.lsstcorp.org/secure/Tests.jspa\#/testCase/LVV-T185}{LVV-T185} &
Jeff Kantor & Draft & 1 & false & Inspection
\\ \hline
\end{longtable}

\textbf{Objective:} \\
Via monitoring and ~comparing with at least 6 months of historical data
on the link, verify that the mean time between failures is less than
summToBaseNetMTBF (90 days) over 1 year.

\textbf{Precondition:} \\
PMCS DMTC-7400-2400 Complete.\\
6 months of historical availability data for this link is available.\\
perSonar installed in Summit and publishing statistics to MadDash.\\
As-built documentation for all of the above is
available.\\[2\baselineskip]

\textbf{Predecessors:} \\
See pre-conditions.

\textbf{Test Personnel:} \\
Ron Lambert (LSST)

\textbf{Test Procedure}
    \begin{longtable}[]{p{1.3cm}p{2cm}p{13cm}}
    %\toprule
    Step & \multicolumn{2}{@{}l}{Description, Input Data and Expected Result} \\ \toprule
    \endhead
            \multirow{3}{*}{\parbox{1.3cm}{ 1
} }
& {\small Description} &
\begin{minipage}[t]{13cm}{\scriptsize
Monitor summit to base networking for at least 1 week

\vspace{\dp0}
} \end{minipage} \\ \cdashline{2-3}
& {\small Test Data} &
\begin{minipage}[t]{13cm}{\scriptsize
LATISS, ComCAM, and/or Full Camera data.

\vspace{\dp0}
} \end{minipage}
\\ \cdashline{2-3}
& {\small Expected Result} &
\begin{minipage}[t]{13cm}{\scriptsize
Summit - base network is operational for 1 week and monitoring data is
collected.

\vspace{\dp0}
} \end{minipage}
\\ \hdashline
        \\ \midrule
            \multirow{3}{*}{\parbox{1.3cm}{ 2
} }
& {\small Description} &
\begin{minipage}[t]{13cm}{\scriptsize
Extrapolate annual availability, compare with at least 6 months of
historical data on the link.

\vspace{\dp0}
} \end{minipage} \\ \cdashline{2-3}
& {\small Test Data} &
\begin{minipage}[t]{13cm}{\scriptsize
Historical and current logs

\vspace{\dp0}
} \end{minipage}
\\ \cdashline{2-3}
& {\small Expected Result} &
\begin{minipage}[t]{13cm}{\scriptsize
The mean time between failures (MTBF) is projected to be less than
summToBaseNetMTBF (90 days) over 1 year.

\vspace{\dp0}
} \end{minipage}
\\ \hdashline
        \\ \midrule
    \end{longtable}


\subsection{[LVV-75] DMS-REQ-0173-V-01: Summit to Base Network Reliability }\label{lvv-75}

\begin{longtable}{ccccc}
\hline
\textbf{Jira Link} & \textbf{Assignee} & \textbf{Status} & \textbf{Priority} & \textbf{Test Cases}\\ \hline
\href{https://jira.lsstcorp.org/browse/LVV-75}{LVV-75} &
Robert Gruendl & Not Covered & 1a &
\begin{tabular}{c}
LVV-T186 \\
\end{tabular}
\\
\hline
\end{longtable}

\textbf{Verification Element Description:} \\
Disconnect, reconnect and recover transfer of data between summit and
base. After disconnecting fiber at an intermediate location between
summit and base, demonstrate reconnection and recovery to transfer of
data at or exceeding rates specified in \citeds{LDM-142} within MTTR
specification. Â~Network operator will provide MTTR data on links during
commissioning and operations.

{\footnotesize
\begin{longtable}{p{3cm}p{13cm}}
\hline
\multicolumn{2}{c}{\textbf{Upstream Requirements}}\\ \hline
Requirement ID & DMS-REQ-0173 \\ \cdashline{1-2}
Requirement Description & \textbf{Specification:} The Summit to Base communications shall be
highly reliable, with Mean Time to Repair (MTTR) \textless{}
\textbf{summToBaseNetMTTR}. \\ \cdashline{1-2}
Requirement Parameters & \textbf{summToBaseNetMTTR = 24{{[}hour{]}}} Mean time to repair,
measured over a 1-yr period. \\ \cdashline{1-2}
Requirement Priority & 1b \\ \cdashline{1-2}
Upper Level Requirement &
\begin{tabular}{cl}
OSS-REQ-0373 & Unscheduled Downtime Subsystem Allocations \\
DMS-REQ-0161 & Optimization of Cost, Reliability and Availability in Order \\
\end{tabular}
\\ \hline
\end{longtable}
}

\subsubsection{[LVV-T186] Verify implementation of Summit to Base Network Reliability }

\begin{longtable}{cccccc}
\hline
\multicolumn{6}{c}{\textbf{Test Case Summary}} \\ \hline
\textbf{Jira Link} & \textbf{Owner} & \textbf{Status} & \textbf{Version} & \textbf{Critical Event} &
\textbf{Verification Type} \\ \hline
\href{https://jira.lsstcorp.org/secure/Tests.jspa\#/testCase/LVV-T186}{LVV-T186} &
Jeff Kantor & Draft & 1 & false & Demonstration
\\ \hline
\end{longtable}

\textbf{Objective:} \\
After disconnecting fiber at an endpoint location on the base side of
the ~summit - base fiber, demonstrate reconnection and recovery to
transfer of data at or exceeding rates specified in \citeds{LDM-142} within MTTR
specification. Network operator will provide MTTR data on links during
commissioning and operations.\\[2\baselineskip]

\textbf{Precondition:} \\
PMCS DMTC-7400-2400 Complete\\
As-built documentation for Summit - Base Network is available.

\textbf{Predecessors:} \\
See pre-conditions.

\textbf{Test Personnel:} \\
Ron Lambert (LSST), Guido Maulen (LSST)

\textbf{Test Procedure}
    \begin{longtable}[]{p{1.3cm}p{2cm}p{13cm}}
    %\toprule
    Step & \multicolumn{2}{@{}l}{Description, Input Data and Expected Result} \\ \toprule
    \endhead
            \multirow{3}{*}{\parbox{1.3cm}{ 1
} }
& {\small Description} &
\begin{minipage}[t]{13cm}{\scriptsize
Disconnect fiber cable at an endpoint location on the base side of the
Summit - Base fiber.

\vspace{\dp0}
} \end{minipage} \\ \cdashline{2-3}
& {\small Test Data} &
\begin{minipage}[t]{13cm}{\scriptsize
LATISS, ComCAM, or FullCam data

\vspace{\dp0}
} \end{minipage}
\\ \cdashline{2-3}
& {\small Expected Result} &
\begin{minipage}[t]{13cm}{\scriptsize
Fiber is disconnected and the fault is detected by the network
monitoring system.

\vspace{\dp0}
} \end{minipage}
\\ \hdashline
        \\ \midrule
            \multirow{3}{*}{\parbox{1.3cm}{ 2
} }
& {\small Description} &
\begin{minipage}[t]{13cm}{\scriptsize
Measure the cable with the OTDR to locate the distance from the end
point. Diagnose that it is a break.

\vspace{\dp0}
} \end{minipage} \\ \cdashline{2-3}
& {\small Test Data} &
\begin{minipage}[t]{13cm}{\scriptsize
NA

\vspace{\dp0}
} \end{minipage}
\\ \cdashline{2-3}
& {\small Expected Result} &
\begin{minipage}[t]{13cm}{\scriptsize
OTDR shows the fiber is disconnected (break).

\vspace{\dp0}
} \end{minipage}
\\ \hdashline
        \\ \midrule
            \multirow{3}{*}{\parbox{1.3cm}{ 3
} }
& {\small Description} &
\begin{minipage}[t]{13cm}{\scriptsize
Elapse time to simulate the following:

\begin{itemize}
\tightlist
\item
  Go to the most inaccessible place which would mean carrying all the
  tools/splicer/generator/tent equipment some ~metres.
\item
  Erect a tent to make the splice
\item
  Start the generator
\item
  Do a splice on some random piece of cable
\item
  At an end point measure the cable again to ensure it is break free.
\item
  Take down and reinstall an isolated pole (not in the actual fiber
  path)
\item
  Put the cable on the pole.
\end{itemize}

\vspace{\dp0}
} \end{minipage} \\ \cdashline{2-3}
& {\small Test Data} &
\begin{minipage}[t]{13cm}{\scriptsize
NA

\vspace{\dp0}
} \end{minipage}
\\ \cdashline{2-3}
& {\small Expected Result} &
\begin{minipage}[t]{13cm}{\scriptsize
Wall clock advances by 24 hours.

\vspace{\dp0}
} \end{minipage}
\\ \hdashline
        \\ \midrule
            \multirow{3}{*}{\parbox{1.3cm}{ 4
} }
& {\small Description} &
\begin{minipage}[t]{13cm}{\scriptsize
Clean fiber connections. ~Restore connection (e.g. reconnect cable).
~Cycle equipment as necessary to confirm fiber is connected.

\vspace{\dp0}
} \end{minipage} \\ \cdashline{2-3}
& {\small Test Data} &
\begin{minipage}[t]{13cm}{\scriptsize
NA

\vspace{\dp0}
} \end{minipage}
\\ \cdashline{2-3}
& {\small Expected Result} &
\begin{minipage}[t]{13cm}{\scriptsize
Network recovers and resumes sending data.

\vspace{\dp0}
} \end{minipage}
\\ \hdashline
        \\ \midrule
            \multirow{3}{*}{\parbox{1.3cm}{ 5
} }
& {\small Description} &
\begin{minipage}[t]{13cm}{\scriptsize
Measure with OTDR to ensure back to normal state.

\vspace{\dp0}
} \end{minipage} \\ \cdashline{2-3}
& {\small Test Data} &
\begin{minipage}[t]{13cm}{\scriptsize
NA

\vspace{\dp0}
} \end{minipage}
\\ \cdashline{2-3}
& {\small Expected Result} &
\begin{minipage}[t]{13cm}{\scriptsize
OTDR indicates normal state.

\vspace{\dp0}
} \end{minipage}
\\ \hdashline
        \\ \midrule
    \end{longtable}


\subsection{[LVV-76] DMS-REQ-0174-V-01: Summit to Base Network Secondary Link }\label{lvv-76}

\begin{longtable}{ccccc}
\hline
\textbf{Jira Link} & \textbf{Assignee} & \textbf{Status} & \textbf{Priority} & \textbf{Test Cases}\\ \hline
\href{https://jira.lsstcorp.org/browse/LVV-76}{LVV-76} &
Robert Gruendl & Not Covered & 1a &
\begin{tabular}{c}
LVV-T187 \\
\end{tabular}
\\
\hline
\end{longtable}

\textbf{Verification Element Description:} \\
This requirement is verified by demonstrating use of a secondary
transfer method (redundant fiber network, microwave link, or
transportable medium) between Summit and Base capable of transferring 1
night of raw data (nCalibExpDay + nRawExpNightMax = 450 + 2800 = 3250
exposures) within summToBaseNet2TransMax (72 hours).

{\footnotesize
\begin{longtable}{p{3cm}p{13cm}}
\hline
\multicolumn{2}{c}{\textbf{Upstream Requirements}}\\ \hline
Requirement ID & DMS-REQ-0174 \\ \cdashline{1-2}
Requirement Description & \textbf{Specification:} The Summit to Base communications shall provide
at least one secondary link or transport mechanism for minimal
operations support in the event of extended outage. This link may
include redundant fiber optics, microwaves, or transportable media. It
shall be capable of transferring one night's worth of raw data in
\textbf{summToBaseNet2TransMax} or less. \\ \cdashline{1-2}
Requirement Parameters & \textbf{summToBaseNet2TransMax = 72{{[}hour{]}}} Maximum time to
transfer one night of data via the network secondary link. \\ \cdashline{1-2}
Requirement Priority & 1b \\ \cdashline{1-2}
Upper Level Requirement &
\begin{tabular}{cl}
DMS-REQ-0173 & Summit to Base Network Reliability \\
OSS-REQ-0049 & Degraded Operational States \\
DMS-REQ-0172 & Summit to Base Network Availability \\
\end{tabular}
\\ \hline
\end{longtable}
}

\subsubsection{[LVV-T187] Verify implementation of Summit to Base Network Secondary Link }

\begin{longtable}{cccccc}
\hline
\multicolumn{6}{c}{\textbf{Test Case Summary}} \\ \hline
\textbf{Jira Link} & \textbf{Owner} & \textbf{Status} & \textbf{Version} & \textbf{Critical Event} &
\textbf{Verification Type} \\ \hline
\href{https://jira.lsstcorp.org/secure/Tests.jspa\#/testCase/LVV-T187}{LVV-T187} &
Jeff Kantor & Draft & 1 & false & Test
\\ \hline
\end{longtable}

\textbf{Objective:} \\
Verify automated fail-over from primary to secondary equipment in Rubin
Observatory DWDM on simulated failure of primary. ~Verify bandwidth
sufficiency on secondary. ~Verify automated recovery to primary
equipment on simulated restoration of primary. ~Repeat for failure of
Rubin Observatory fiber and fail-over to AURA fiber and DWDM.
~Demonstrate use of secondary in ``catch-up'' mode.

\textbf{Precondition:} \\
PMCS DMTC-7400-2400 complete.\\
As-built documentation for Summit - Base Network is available.

\textbf{Predecessors:} \\
See pre-conditions.

\textbf{Test Personnel:} \\
Ron Lambert (LSST)

\textbf{Test Procedure}
    \begin{longtable}[]{p{1.3cm}p{2cm}p{13cm}}
    %\toprule
    Step & \multicolumn{2}{@{}l}{Description, Input Data and Expected Result} \\ \toprule
    \endhead
            \multirow{3}{*}{\parbox{1.3cm}{ 1
} }
& {\small Description} &
\begin{minipage}[t]{13cm}{\scriptsize
Transfer data between summit and base on primary equipment (LSST Summit
- Base) over uninterrupted 1 day period. ~

\vspace{\dp0}
} \end{minipage} \\ \cdashline{2-3}
& {\small Test Data} &
\begin{minipage}[t]{13cm}{\scriptsize
LATISS, ComCAM, or FullCAM data.

\vspace{\dp0}
} \end{minipage}
\\ \cdashline{2-3}
& {\small Expected Result} &
\begin{minipage}[t]{13cm}{\scriptsize
Normal operations.

\vspace{\dp0}
} \end{minipage}
\\ \hdashline
        \\ \midrule
            \multirow{3}{*}{\parbox{1.3cm}{ 2
} }
& {\small Description} &
\begin{minipage}[t]{13cm}{\scriptsize
Simulate equipment outage by disconnecting power card from primary DWDM
equipment on base side of Summit - Base Fiber.

\vspace{\dp0}
} \end{minipage} \\ \cdashline{2-3}
& {\small Test Data} &
\begin{minipage}[t]{13cm}{\scriptsize
NA

\vspace{\dp0}
} \end{minipage}
\\ \cdashline{2-3}
& {\small Expected Result} &
\begin{minipage}[t]{13cm}{\scriptsize
Network fails over to secondary equipment in \textless{}=60s.

\vspace{\dp0}
} \end{minipage}
\\ \hdashline
        \\ \midrule
            \multirow{3}{*}{\parbox{1.3cm}{ 3
} }
& {\small Description} &
\begin{minipage}[t]{13cm}{\scriptsize
Transfer data between summit and base over secondary equipment
uninterrupted 1 day period while monitoring network.

\vspace{\dp0}
} \end{minipage} \\ \cdashline{2-3}
& {\small Test Data} &
\begin{minipage}[t]{13cm}{\scriptsize
NA

\vspace{\dp0}
} \end{minipage}
\\ \cdashline{2-3}
& {\small Expected Result} &
\begin{minipage}[t]{13cm}{\scriptsize
Verify that secondary equipment is capable of transferring 1 night of
raw data (nCalibExpDay + nRawExpNightMax = 450 + 2800 = ~3250 exposures)
within summToBaseNet2TransMax (72 hours), i.e. at or exceeding rates
specified in LDM-142.

\vspace{\dp0}
} \end{minipage}
\\ \hdashline
        \\ \midrule
            \multirow{3}{*}{\parbox{1.3cm}{ 4
} }
& {\small Description} &
\begin{minipage}[t]{13cm}{\scriptsize
Restore ~primary equipment (i.e. reconnect power card to primary
equipment.)

\vspace{\dp0}
} \end{minipage} \\ \cdashline{2-3}
& {\small Test Data} &
\begin{minipage}[t]{13cm}{\scriptsize
NA

\vspace{\dp0}
} \end{minipage}
\\ \cdashline{2-3}
& {\small Expected Result} &
\begin{minipage}[t]{13cm}{\scriptsize
Network recovers to primary in \textless{}= 60s.

\vspace{\dp0}
} \end{minipage}
\\ \hdashline
        \\ \midrule
            \multirow{3}{*}{\parbox{1.3cm}{ 5
} }
& {\small Description} &
\begin{minipage}[t]{13cm}{\scriptsize
Simulate fiber outage by disconnecting fiber from primary DWDM equipment
on base side of Summit - Base Fiber.

\vspace{\dp0}
} \end{minipage} \\ \cdashline{2-3}
& {\small Test Data} &
\begin{minipage}[t]{13cm}{\scriptsize
NA

\vspace{\dp0}
} \end{minipage}
\\ \cdashline{2-3}
& {\small Expected Result} &
\begin{minipage}[t]{13cm}{\scriptsize
Network fails over to AURA DWDM and fiber.

\vspace{\dp0}
} \end{minipage}
\\ \hdashline
        \\ \midrule
            \multirow{3}{*}{\parbox{1.3cm}{ 6
} }
& {\small Description} &
\begin{minipage}[t]{13cm}{\scriptsize
Transfer data between summit and base over AURA fiber and equipment
uninterrupted 1 day period while monitoring network.

\vspace{\dp0}
} \end{minipage} \\ \cdashline{2-3}
& {\small Test Data} &
\begin{minipage}[t]{13cm}{\scriptsize
LATISS, ComCAM, or FullCAM data.

\vspace{\dp0}
} \end{minipage}
\\ \cdashline{2-3}
& {\small Expected Result} &
\begin{minipage}[t]{13cm}{\scriptsize
Verify that AURA fiber and equipment is capable of transferring 1 night
of raw data (nCalibExpDay + nRawExpNightMax = 450 + 2800 = ~3250
exposures) within summToBaseNet2TransMax (72 hours), i.e. at or
exceeding rates specified in LDM-142.

\vspace{\dp0}
} \end{minipage}
\\ \hdashline
        \\ \midrule
            \multirow{3}{*}{\parbox{1.3cm}{ 7
} }
& {\small Description} &
\begin{minipage}[t]{13cm}{\scriptsize
Restore ~primary fiber (i.e. reconnect fiber to Rubin Observatory DWDM
equipment.)

\vspace{\dp0}
} \end{minipage} \\ \cdashline{2-3}
& {\small Test Data} &
\\ \cdashline{2-3}
& {\small Expected Result} &
\begin{minipage}[t]{13cm}{\scriptsize
Network recovers to Rubin Observatory fiber and DWDM.

\vspace{\dp0}
} \end{minipage}
\\ \hdashline
        \\ \midrule
            \multirow{3}{*}{\parbox{1.3cm}{ 8
} }
& {\small Description} &
\begin{minipage}[t]{13cm}{\scriptsize
Demonstrate use of secondary in ``catch-up'' mode.

\vspace{\dp0}
} \end{minipage} \\ \cdashline{2-3}
& {\small Test Data} &
\begin{minipage}[t]{13cm}{\scriptsize
DAQ data buffer full of images and associated meta-data

\vspace{\dp0}
} \end{minipage}
\\ \cdashline{2-3}
& {\small Expected Result} &
\begin{minipage}[t]{13cm}{\scriptsize
Images from DAQ buffer and associated metadata are retrievable over
secondary path while current observing data is being transferred over
primary path.

\vspace{\dp0}
} \end{minipage}
\\ \hdashline
        \\ \midrule
    \end{longtable}


\subsection{[LVV-77] DMS-REQ-0175-V-01: Summit to Base Network Ownership and Operation }\label{lvv-77}

\begin{longtable}{ccccc}
\hline
\textbf{Jira Link} & \textbf{Assignee} & \textbf{Status} & \textbf{Priority} & \textbf{Test Cases}\\ \hline
\href{https://jira.lsstcorp.org/browse/LVV-77}{LVV-77} &
Robert Gruendl & Not Covered & 1a &
\begin{tabular}{c}
LVV-T188 \\
\end{tabular}
\\
\hline
\end{longtable}

\textbf{Verification Element Description:} \\
This requirement is verified by inspecting construction and operations
contracts and Indefeasible Rights to Use (IRUs).

{\footnotesize
\begin{longtable}{p{3cm}p{13cm}}
\hline
\multicolumn{2}{c}{\textbf{Upstream Requirements}}\\ \hline
Requirement ID & DMS-REQ-0175 \\ \cdashline{1-2}
Requirement Description & \textbf{Specification:} The Summit to Base communications link shall be
owned and operated by LSST and/or the operations entity to ensure
responsiveness of support. \\ \cdashline{1-2}
Requirement Priority & 1b \\ \cdashline{1-2}
Upper Level Requirement &
\begin{tabular}{cl}
DMS-REQ-0173 & Summit to Base Network Reliability \\
OSS-REQ-0036 & Local Autonomous Administration of System Sites \\
DMS-REQ-0172 & Summit to Base Network Availability \\
\end{tabular}
\\ \hline
\end{longtable}
}

\subsubsection{[LVV-T188] Verify implementation of Summit to Base Network Ownership and Operation }

\begin{longtable}{cccccc}
\hline
\multicolumn{6}{c}{\textbf{Test Case Summary}} \\ \hline
\textbf{Jira Link} & \textbf{Owner} & \textbf{Status} & \textbf{Version} & \textbf{Critical Event} &
\textbf{Verification Type} \\ \hline
\href{https://jira.lsstcorp.org/secure/Tests.jspa\#/testCase/LVV-T188}{LVV-T188} &
Jeff Kantor & Draft & 1 & false & Inspection
\\ \hline
\end{longtable}

\textbf{Objective:} \\
Inspect construction and operations contracts and Indefeasible Rights to
Use (IRUs).

\textbf{Precondition:} \\
As-built documentation for all of the above contracts and IRUs is
available.

\textbf{Predecessors:} \\
PMCS DMTC-7400-2140, -2240, -2330 Complete

\textbf{Test Personnel:} \\
Jeff Kantor (LSST)

\textbf{Test Procedure}
    \begin{longtable}[]{p{1.3cm}p{2cm}p{13cm}}
    %\toprule
    Step & \multicolumn{2}{@{}l}{Description, Input Data and Expected Result} \\ \toprule
    \endhead
            \multirow{3}{*}{\parbox{1.3cm}{ 1
} }
& {\small Description} &
\begin{minipage}[t]{13cm}{\scriptsize
Examine contracts with REUNA and telefonica for fiber ownership and
maintenance terms.

\vspace{\dp0}
} \end{minipage} \\ \cdashline{2-3}
& {\small Test Data} &
\\ \cdashline{2-3}
& {\small Expected Result} &
\begin{minipage}[t]{13cm}{\scriptsize
Rubin Observatory is owner of fibers on AURA property and Summit - Base
DWDM ~and has 15-year IRU for use of fibers on all segments. ~REUNA is
owner of LS - SCL DWDM on AURA property and in Santiago, and is operator
on all fibers and DWDM. ~Telefonica is contracted to maintain fibers not
on AURA property.

\vspace{\dp0}
} \end{minipage}
\\ \hdashline
        \\ \midrule
    \end{longtable}


\subsection{[LVV-81] DMS-REQ-0180-V-01: Base to Archive Network }\label{lvv-81}

\begin{longtable}{ccccc}
\hline
\textbf{Jira Link} & \textbf{Assignee} & \textbf{Status} & \textbf{Priority} & \textbf{Test Cases}\\ \hline
\href{https://jira.lsstcorp.org/browse/LVV-81}{LVV-81} &
Robert Gruendl & Not Covered & 1a &
\begin{tabular}{c}
LVV-T193 \\
\end{tabular}
\\
\hline
\end{longtable}

\textbf{Verification Element Description:} \\
This requirement is verified by transferring simulated or pre-cursor
image data and meta-data between base and archive over an uninterrupted
1 day period. Â~Analyze the network performance and show that data can
be transferred by DAQ within the required time.

{\footnotesize
\begin{longtable}{p{3cm}p{13cm}}
\hline
\multicolumn{2}{c}{\textbf{Upstream Requirements}}\\ \hline
Requirement ID & DMS-REQ-0180 \\ \cdashline{1-2}
Requirement Description & \textbf{Specification:} The DMS shall provide communications
infrastructure between the Base Facility and the Archive Center
sufficient to carry scientific data and associated metadata for each
image in no more than time \textbf{baseToArchiveMaxTransferTime}. \\ \cdashline{1-2}
Requirement Parameters & \textbf{baseToArchiveMaxTransferTime = 5{{[}second{]}}} Maximum time
interval to transfer a full Crosstalk Corrected Exposure and all related
metadata from the Base Facility to the Archive Center. \\ \cdashline{1-2}
Requirement Priority & 1b \\ \cdashline{1-2}
Upper Level Requirement &
\begin{tabular}{cl}
OSS-REQ-0053 & Base-Archive Connectivity Loss \\
OSS-REQ-0055 & Base Updating from Archive \\
DMS-REQ-0162 & Pipeline Throughput \\
\end{tabular}
\\ \hline
\end{longtable}
}

\subsubsection{[LVV-T193] Verify implementation of Base to Archive Network }

\begin{longtable}{cccccc}
\hline
\multicolumn{6}{c}{\textbf{Test Case Summary}} \\ \hline
\textbf{Jira Link} & \textbf{Owner} & \textbf{Status} & \textbf{Version} & \textbf{Critical Event} &
\textbf{Verification Type} \\ \hline
\href{https://jira.lsstcorp.org/secure/Tests.jspa\#/testCase/LVV-T193}{LVV-T193} &
Jeff Kantor & Draft & 1 & false & Test
\\ \hline
\end{longtable}

\textbf{Objective:} \\
Analyze the network and show that data acquired by a DAQ can be
transferred within the required time, i.e. verify that link is capable
of transferring image for prompt processing in oArchiveMaxTransferTime =
5{[}second{]}, i.e. at or exceeding rates specified in \citeds{LDM-142}.

\textbf{Precondition:} \\
Archiver/Forwarders are configured at Base, connected to REUNA DWDM,
loaded with simulated or pre-cursor data.\\
Archiver/Forwarder receivers or other capability is on configured at
LDF, connected to Base - Archive Network.\\
As-built documentation for all of the above is available.

\textbf{Predecessors:} \\
PMCS DM-Net-5 Complete

\textbf{Test Personnel:} \\
Ron Lambert (LSST)

\textbf{Test Procedure}
    \begin{longtable}[]{p{1.3cm}p{2cm}p{13cm}}
    %\toprule
    Step & \multicolumn{2}{@{}l}{Description, Input Data and Expected Result} \\ \toprule
    \endhead
            \multirow{3}{*}{\parbox{1.3cm}{ 1
} }
& {\small Description} &
\begin{minipage}[t]{13cm}{\scriptsize
Transfer data between base and archive while monitoring the network over
uninterrupted 1 day period (with repeated transfers on normal observing
cadence).

\vspace{\dp0}
} \end{minipage} \\ \cdashline{2-3}
& {\small Test Data} &
\begin{minipage}[t]{13cm}{\scriptsize
LATISS, ComCAM, or FullCAM data.

\vspace{\dp0}
} \end{minipage}
\\ \cdashline{2-3}
& {\small Expected Result} &
\begin{minipage}[t]{13cm}{\scriptsize
Data transfers occur without significant delay or frequent latency
spikes.

\vspace{\dp0}
} \end{minipage}
\\ \hdashline
        \\ \midrule
            \multirow{3}{*}{\parbox{1.3cm}{ 2
} }
& {\small Description} &
\begin{minipage}[t]{13cm}{\scriptsize
~Analyze the network logs and monitoring system to determine average and
peak latency and packet loss statistics.

\vspace{\dp0}
} \end{minipage} \\ \cdashline{2-3}
& {\small Test Data} &
\\ \cdashline{2-3}
& {\small Expected Result} &
\begin{minipage}[t]{13cm}{\scriptsize
Data can be transferred within the required time, i.e. verify that link
is capable of transferring image for prompt processing in
oArchiveMaxTransferTime = 5{[}second{]}. Verify transfer of data at or
exceeding rates specified in LDM-142 at least 98\% of the time.

\vspace{\dp0}
} \end{minipage}
\\ \hdashline
        \\ \midrule
    \end{longtable}


\subsection{[LVV-82] DMS-REQ-0181-V-01: Base to Archive Network Availability }\label{lvv-82}

\begin{longtable}{ccccc}
\hline
\textbf{Jira Link} & \textbf{Assignee} & \textbf{Status} & \textbf{Priority} & \textbf{Test Cases}\\ \hline
\href{https://jira.lsstcorp.org/browse/LVV-82}{LVV-82} &
Robert Gruendl & Not Covered & 1a &
\begin{tabular}{c}
LVV-T194 \\
\end{tabular}
\\
\hline
\end{longtable}

\textbf{Verification Element Description:} \\
This requirement is verified by transferring data between base and
archive over uninterrupted 1 week period, modeling to extrapolate to an
annual failure rate, and verifying that is within the requirement.

{\footnotesize
\begin{longtable}{p{3cm}p{13cm}}
\hline
\multicolumn{2}{c}{\textbf{Upstream Requirements}}\\ \hline
Requirement ID & DMS-REQ-0181 \\ \cdashline{1-2}
Requirement Description & \textbf{Specification:} The Base to Archive communications shall be
highly available, with MTBF \textgreater{} \textbf{baseToArchNetMTBF}. \\ \cdashline{1-2}
Requirement Parameters & \textbf{baseToArchNetMTBF = 180{{[}day{]}}} Mean time between failures,
measured over a 1-yr period. \\ \cdashline{1-2}
Requirement Priority & 1b \\ \cdashline{1-2}
Upper Level Requirement &
\begin{tabular}{cl}
OSS-REQ-0053 & Base-Archive Connectivity Loss \\
DMS-REQ-0162 & Pipeline Throughput \\
DMS-REQ-0161 & Optimization of Cost, Reliability and Availability in Order \\
\end{tabular}
\\ \hline
\end{longtable}
}

\subsubsection{[LVV-T194] Verify implementation of Base to Archive Network Availability }

\begin{longtable}{cccccc}
\hline
\multicolumn{6}{c}{\textbf{Test Case Summary}} \\ \hline
\textbf{Jira Link} & \textbf{Owner} & \textbf{Status} & \textbf{Version} & \textbf{Critical Event} &
\textbf{Verification Type} \\ \hline
\href{https://jira.lsstcorp.org/secure/Tests.jspa\#/testCase/LVV-T194}{LVV-T194} &
Jeff Kantor & Draft & 1 & false & Test
\\ \hline
\end{longtable}

\textbf{Objective:} \\
From test and historical data, extrapolate to a full year to estimate if
expect to meet baseToArchNetMTBF = 180{[}day{]}. ~Note that this is for
complete loss of transfer service (simultaneous failure on all paths),
not a single path failure with successful fail-over.

\textbf{Precondition:} \\
Archiver/Forwarders are configured at Base, connected to REUNA DWDM,
loaded with simulated or pre-cursor data.\\
Archiver/Forwarder receivers or other capability is on configured at
LDF, connected to Base - Archive Network.\\
At least 6 months of historical monitoring data on this link is
available.\\
As-built documentation for all of the above is available.

\textbf{Predecessors:} \\
~PMCS DMTC-7400-2130 Complete

\textbf{Test Personnel:} \\


\textbf{Test Procedure}
    \begin{longtable}[]{p{1.3cm}p{2cm}p{13cm}}
    %\toprule
    Step & \multicolumn{2}{@{}l}{Description, Input Data and Expected Result} \\ \toprule
    \endhead
            \multirow{3}{*}{\parbox{1.3cm}{ 1
} }
& {\small Description} &
\begin{minipage}[t]{13cm}{\scriptsize
Transfer data between base and archive over uninterrupted 1 week period.

\vspace{\dp0}
} \end{minipage} \\ \cdashline{2-3}
& {\small Test Data} &
\begin{minipage}[t]{13cm}{\scriptsize
LATISS, ComCAM, or FullCAM data.

\vspace{\dp0}
} \end{minipage}
\\ \cdashline{2-3}
& {\small Expected Result} &
\begin{minipage}[t]{13cm}{\scriptsize
Data is successfully transferred during the entire week.

\vspace{\dp0}
} \end{minipage}
\\ \hdashline
        \\ \midrule
            \multirow{3}{*}{\parbox{1.3cm}{ 2
} }
& {\small Description} &
\begin{minipage}[t]{13cm}{\scriptsize
Analyze monitoring/performance data, compare to historical data, and
extrapolate to a full year, average and peak throughput and latency.

\vspace{\dp0}
} \end{minipage} \\ \cdashline{2-3}
& {\small Test Data} &
\begin{minipage}[t]{13cm}{\scriptsize
NA

\vspace{\dp0}
} \end{minipage}
\\ \cdashline{2-3}
& {\small Expected Result} &
\begin{minipage}[t]{13cm}{\scriptsize
Extrapolated network availability meets baseToArchNetMTBF =
180{[}day{]}. ~Note that this is for complete loss of transfer service
(all paths), not a single path failure with successful fail-over.

\vspace{\dp0}
} \end{minipage}
\\ \hdashline
        \\ \midrule
    \end{longtable}


\subsection{[LVV-83] DMS-REQ-0182-V-01: Base to Archive Network Reliability }\label{lvv-83}

\begin{longtable}{ccccc}
\hline
\textbf{Jira Link} & \textbf{Assignee} & \textbf{Status} & \textbf{Priority} & \textbf{Test Cases}\\ \hline
\href{https://jira.lsstcorp.org/browse/LVV-83}{LVV-83} &
Robert Gruendl & Not Covered & 1a &
\begin{tabular}{c}
LVV-T195 \\
\end{tabular}
\\
\hline
\end{longtable}

\textbf{Verification Element Description:} \\
Disconnect, reconnect and recover transfer of data between summit and
base, after disconnecting fiber at an intermediate location between base
and archive

{\footnotesize
\begin{longtable}{p{3cm}p{13cm}}
\hline
\multicolumn{2}{c}{\textbf{Upstream Requirements}}\\ \hline
Requirement ID & DMS-REQ-0182 \\ \cdashline{1-2}
Requirement Description & \textbf{Specification:} The Base to Archive communications shall be
highly reliable, with MTTR \textless{} \textbf{baseToArchNetMTTR}. \\ \cdashline{1-2}
Requirement Parameters & \textbf{baseToArchNetMTTR = 48{{[}hour{]}}} Mean time to repair,
measured over a 1-yr period. \\ \cdashline{1-2}
Requirement Priority & 1b \\ \cdashline{1-2}
Upper Level Requirement &
\begin{tabular}{cl}
OSS-REQ-0053 & Base-Archive Connectivity Loss \\
DMS-REQ-0161 & Optimization of Cost, Reliability and Availability in Order \\
\end{tabular}
\\ \hline
\end{longtable}
}

\subsubsection{[LVV-T195] Verify implementation of Base to Archive Network Reliability }

\begin{longtable}{cccccc}
\hline
\multicolumn{6}{c}{\textbf{Test Case Summary}} \\ \hline
\textbf{Jira Link} & \textbf{Owner} & \textbf{Status} & \textbf{Version} & \textbf{Critical Event} &
\textbf{Verification Type} \\ \hline
\href{https://jira.lsstcorp.org/secure/Tests.jspa\#/testCase/LVV-T195}{LVV-T195} &
Jeff Kantor & Draft & 1 & false & Test
\\ \hline
\end{longtable}

\textbf{Objective:} \\
After disconnecting fiber at a location in the Base - Archive, network,
verify recovery can occur within baseToArchNetMTTR = 48{[}hour{]}.

\textbf{Precondition:} \\
Archiver/Forwarders are configured at Base, connected to REUNA DWDM,
loaded with simulated or pre-cursor data.\\
Archiver/Forwarder receivers or other capability is on configured at
LDF, connected to Base - Archive Network.\\
At least 6 months of monitoring data for this link is available.\\
As-built documentation for all of the above is available.

\textbf{Predecessors:} \\
PMCS DM-NET-5 Complete

\textbf{Test Personnel:} \\
Ron Lambert (LSST), Albert Astudillo (REUNA), Jeronimo Bezerra
(FIU/AmLight), Matt Kollross (NCSA)

\textbf{Test Procedure}
    \begin{longtable}[]{p{1.3cm}p{2cm}p{13cm}}
    %\toprule
    Step & \multicolumn{2}{@{}l}{Description, Input Data and Expected Result} \\ \toprule
    \endhead
            \multirow{3}{*}{\parbox{1.3cm}{ 1
} }
& {\small Description} &
\begin{minipage}[t]{13cm}{\scriptsize
Disconnect primary fiber on base side of Base - ~Archive network.

\vspace{\dp0}
} \end{minipage} \\ \cdashline{2-3}
& {\small Test Data} &
\begin{minipage}[t]{13cm}{\scriptsize
LATISS, ComCAM, or FullCAM data.

\vspace{\dp0}
} \end{minipage}
\\ \cdashline{2-3}
& {\small Expected Result} &
\begin{minipage}[t]{13cm}{\scriptsize
Network fails over to secondary path.

\vspace{\dp0}
} \end{minipage}
\\ \hdashline
        \\ \midrule
            \multirow{3}{*}{\parbox{1.3cm}{ 2
} }
& {\small Description} &
\begin{minipage}[t]{13cm}{\scriptsize
Simulate diagnosis and repair by elapsed time.\\[2\baselineskip]

\vspace{\dp0}
} \end{minipage} \\ \cdashline{2-3}
& {\small Test Data} &
\begin{minipage}[t]{13cm}{\scriptsize
NA

\vspace{\dp0}
} \end{minipage}
\\ \cdashline{2-3}
& {\small Expected Result} &
\begin{minipage}[t]{13cm}{\scriptsize
Wall clock advances by 48 hours. ~Data is successfully transferred over
secondary path.

\vspace{\dp0}
} \end{minipage}
\\ \hdashline
        \\ \midrule
            \multirow{3}{*}{\parbox{1.3cm}{ 3
} }
& {\small Description} &
\begin{minipage}[t]{13cm}{\scriptsize
Reconnect primary fiber on base side of Base - Archive network.

\vspace{\dp0}
} \end{minipage} \\ \cdashline{2-3}
& {\small Test Data} &
\begin{minipage}[t]{13cm}{\scriptsize
NA

\vspace{\dp0}
} \end{minipage}
\\ \cdashline{2-3}
& {\small Expected Result} &
\begin{minipage}[t]{13cm}{\scriptsize
Network recovers to primary path.~

\vspace{\dp0}
} \end{minipage}
\\ \hdashline
        \\ \midrule
            \multirow{3}{*}{\parbox{1.3cm}{ 4
} }
& {\small Description} &
\begin{minipage}[t]{13cm}{\scriptsize
Analyze fail-over and recovery times. ~Compare to historical data and
extrapolate to MTTR.

\vspace{\dp0}
} \end{minipage} \\ \cdashline{2-3}
& {\small Test Data} &
\\ \cdashline{2-3}
& {\small Expected Result} &
\begin{minipage}[t]{13cm}{\scriptsize
Verify recovery can occur within baseToArchNetMTTR = 48{[}hour{]}.
Demonstrate reconnection and recovery to transfer of data at or
exceeding rates specified in LDM-142.

\vspace{\dp0}
} \end{minipage}
\\ \hdashline
        \\ \midrule
    \end{longtable}


\subsection{[LVV-84] DMS-REQ-0183-V-01: Base to Archive Network Secondary Link }\label{lvv-84}

\begin{longtable}{ccccc}
\hline
\textbf{Jira Link} & \textbf{Assignee} & \textbf{Status} & \textbf{Priority} & \textbf{Test Cases}\\ \hline
\href{https://jira.lsstcorp.org/browse/LVV-84}{LVV-84} &
Robert Gruendl & Not Covered & 1a &
\begin{tabular}{c}
LVV-T196 \\
\end{tabular}
\\
\hline
\end{longtable}

\textbf{Verification Element Description:} \\
This requirement is verified by disconnecting the primary link, failing
over to the secondary link, reconnecting primary link, and failing back
to primary link, while verifying data is transferred within required
times.

{\footnotesize
\begin{longtable}{p{3cm}p{13cm}}
\hline
\multicolumn{2}{c}{\textbf{Upstream Requirements}}\\ \hline
Requirement ID & DMS-REQ-0183 \\ \cdashline{1-2}
Requirement Description & \textbf{Specification:} The Base to Archive communications shall provide
a secondary link or transport mechanism (e.g. protected circuit) for
operations support and ``catch up'' in the event of extended outage
which is capable of transferring data at least the same rate as the
required minimum capacity of the primary link. \\ \cdashline{1-2}
Requirement Priority & 1b \\ \cdashline{1-2}
Upper Level Requirement &
\begin{tabular}{cl}
DMS-REQ-0181 & Base to Archive Network Availability \\
DMS-REQ-0182 & Base to Archive Network Reliability \\
OSS-REQ-0049 & Degraded Operational States \\
\end{tabular}
\\ \hline
\end{longtable}
}

\subsubsection{[LVV-T196] Verify implementation of Base to Archive Network Secondary Link }

\begin{longtable}{cccccc}
\hline
\multicolumn{6}{c}{\textbf{Test Case Summary}} \\ \hline
\textbf{Jira Link} & \textbf{Owner} & \textbf{Status} & \textbf{Version} & \textbf{Critical Event} &
\textbf{Verification Type} \\ \hline
\href{https://jira.lsstcorp.org/secure/Tests.jspa\#/testCase/LVV-T196}{LVV-T196} &
Jeff Kantor & Draft & 1 & false & Test
\\ \hline
\end{longtable}

\textbf{Objective:} \\
~Simulate outage on primary and verify fail over to secondary link and
capacity on secondary link, restore primary link and verify that network
recovers to primary. Demonstrate use of secondary path in ``catch-up''
mode.

\textbf{Precondition:} \\
Archiver/Forwarders are configured at Base, connected to REUNA DWDM,
loaded with simulated or pre-cursor data.\\
Archiver/Forwarder receivers or other capability is on configured at
LDF, connected to Base - Archive Network.\\
As-built documentation for all of the above is available.

\textbf{Predecessors:} \\
PMCS DM-NET-5 Complete\\
PMCS DMTC-8000-0990 Complete\\
PMCS DMTC-8100-2130 Complete\\
PMCS DMTC-8100-2530 Complete\\
PMCS DMTC-8200-0600 Complete

\textbf{Test Personnel:} \\
Ron Lambert (LSST), Albert Astudillo (REUNA), Jeronimo Bezerra
(FIU/AmLight), Matt Kollross (NCSA)

\textbf{Test Procedure}
    \begin{longtable}[]{p{1.3cm}p{2cm}p{13cm}}
    %\toprule
    Step & \multicolumn{2}{@{}l}{Description, Input Data and Expected Result} \\ \toprule
    \endhead
            \multirow{3}{*}{\parbox{1.3cm}{ 1
} }
& {\small Description} &
\begin{minipage}[t]{13cm}{\scriptsize
Transfer data between base and archive on primary links over
uninterrupted 1 day period.

\vspace{\dp0}
} \end{minipage} \\ \cdashline{2-3}
& {\small Test Data} &
\begin{minipage}[t]{13cm}{\scriptsize
LATISS, ComCAM, or FullCAM data.

\vspace{\dp0}
} \end{minipage}
\\ \cdashline{2-3}
& {\small Expected Result} &
\begin{minipage}[t]{13cm}{\scriptsize
Data is successfully transferred over primary link at or exceeding rates
specified in LDM-142 throughout period.

\vspace{\dp0}
} \end{minipage}
\\ \hdashline
        \\ \midrule
            \multirow{3}{*}{\parbox{1.3cm}{ 2
} }
& {\small Description} &
\begin{minipage}[t]{13cm}{\scriptsize
Simulate outage by disconnecting fiber on primary fiber on Base side of
Base - Archive Network.

\vspace{\dp0}
} \end{minipage} \\ \cdashline{2-3}
& {\small Test Data} &
\begin{minipage}[t]{13cm}{\scriptsize
NA

\vspace{\dp0}
} \end{minipage}
\\ \cdashline{2-3}
& {\small Expected Result} &
\begin{minipage}[t]{13cm}{\scriptsize
Network fails over to secondary links in \textless{}=60s

\vspace{\dp0}
} \end{minipage}
\\ \hdashline
        \\ \midrule
            \multirow{3}{*}{\parbox{1.3cm}{ 3
} }
& {\small Description} &
\begin{minipage}[t]{13cm}{\scriptsize
Transfer data between base and archive over secondary equipment
uninterrupted 1 day period.

\vspace{\dp0}
} \end{minipage} \\ \cdashline{2-3}
& {\small Test Data} &
\begin{minipage}[t]{13cm}{\scriptsize
LATISS, ComCAM, or FullCAM data.

\vspace{\dp0}
} \end{minipage}
\\ \cdashline{2-3}
& {\small Expected Result} &
\begin{minipage}[t]{13cm}{\scriptsize
Data is successfully transferred over secondary link ~at or exceeding
rates specified in LDM-142 throughout period.

\vspace{\dp0}
} \end{minipage}
\\ \hdashline
        \\ \midrule
            \multirow{3}{*}{\parbox{1.3cm}{ 4
} }
& {\small Description} &
\begin{minipage}[t]{13cm}{\scriptsize
Restore connection on primary link by reconnecting
fiber.\\[2\baselineskip]

\vspace{\dp0}
} \end{minipage} \\ \cdashline{2-3}
& {\small Test Data} &
\begin{minipage}[t]{13cm}{\scriptsize
NA

\vspace{\dp0}
} \end{minipage}
\\ \cdashline{2-3}
& {\small Expected Result} &
\begin{minipage}[t]{13cm}{\scriptsize
Network recovers to primary.

\vspace{\dp0}
} \end{minipage}
\\ \hdashline
        \\ \midrule
            \multirow{3}{*}{\parbox{1.3cm}{ 5
} }
& {\small Description} &
\begin{minipage}[t]{13cm}{\scriptsize
Demonstrate use of secondary in catch-up mode.

\vspace{\dp0}
} \end{minipage} \\ \cdashline{2-3}
& {\small Test Data} &
\begin{minipage}[t]{13cm}{\scriptsize
DAQ buffer full of images and associated metadata.

\vspace{\dp0}
} \end{minipage}
\\ \cdashline{2-3}
& {\small Expected Result} &
\begin{minipage}[t]{13cm}{\scriptsize
Images from DAQ buffer and associated metadata are retrievable over
secondary path while current observing data is being transferred over
primary path.

\vspace{\dp0}
} \end{minipage}
\\ \hdashline
        \\ \midrule
    \end{longtable}


\subsection{[LVV-88] DMS-REQ-0188-V-01: Archive to Data Access Center Network }\label{lvv-88}

\begin{longtable}{ccccc}
\hline
\textbf{Jira Link} & \textbf{Assignee} & \textbf{Status} & \textbf{Priority} & \textbf{Test Cases}\\ \hline
\href{https://jira.lsstcorp.org/browse/LVV-88}{LVV-88} &
Robert Gruendl & Not Covered & 1a &
\begin{tabular}{c}
LVV-T200 \\
\end{tabular}
\\
\hline
\end{longtable}

\textbf{Verification Element Description:} \\
This requirement is verified by transferring data between archive and
both DACs over uninterrupted 1 day period, analyzing the network
performance, and verifying that data can be transferred within the
required time.

{\footnotesize
\begin{longtable}{p{3cm}p{13cm}}
\hline
\multicolumn{2}{c}{\textbf{Upstream Requirements}}\\ \hline
Requirement ID & DMS-REQ-0188 \\ \cdashline{1-2}
Requirement Description & \textbf{Specification:} The DMS shall provide communications
infrastructure between the Archive Center and Data Access Centers
sufficient to carry scientific data and associated metadata in support
of community and EPO access. Aggregate bandwidth for data transfers from
the Archive Center to Data Centers shall be at least
\textbf{archToDacBandwidth}. \\ \cdashline{1-2}
Requirement Parameters & \textbf{archToDacBandwidth = 10000{{[}megabit per second{]}}} Aggregate
bandwidth capacity for data transfers between the Archive and Data
Access Centers. \\ \cdashline{1-2}
Requirement Priority & 1b \\ \cdashline{1-2}
Upper Level Requirement &
\begin{tabular}{cl}
OSS-REQ-0004 & The Archive Facility \\
\end{tabular}
\\ \hline
\end{longtable}
}

\subsubsection{[LVV-T200] Verify implementation of Archive to Data Access Center Network }

\begin{longtable}{cccccc}
\hline
\multicolumn{6}{c}{\textbf{Test Case Summary}} \\ \hline
\textbf{Jira Link} & \textbf{Owner} & \textbf{Status} & \textbf{Version} & \textbf{Critical Event} &
\textbf{Verification Type} \\ \hline
\href{https://jira.lsstcorp.org/secure/Tests.jspa\#/testCase/LVV-T200}{LVV-T200} &
Jeff Kantor & Draft & 1 & false & Test
\\ \hline
\end{longtable}

\textbf{Objective:} \\
Monitor and analyze data transfers to verify networks can transfer data
at archToDacBandwidth = 10000{[}megabit per second{]}, i.e. at or
exceeding rates specified in \citeds{LDM-142}.

\textbf{Precondition:} \\
Data is staged in LDF and data transfer capabilities to US DAC and
Chilean DAC are in place.\\
At least 6 months of historical monitoring data is available on these
network links.\\
As-built documentation for all of the above is available.

\textbf{Predecessors:} \\
PMCS DMTC-8100-2550 Complete

\textbf{Test Personnel:} \\
Ron Lambert (LSST), Albert Astudillo (REUNA), Jeronimo Bezerra
(FIU/AmLight), Matt Kollross (NCSA)

\textbf{Test Procedure}
    \begin{longtable}[]{p{1.3cm}p{2cm}p{13cm}}
    %\toprule
    Step & \multicolumn{2}{@{}l}{Description, Input Data and Expected Result} \\ \toprule
    \endhead
            \multirow{3}{*}{\parbox{1.3cm}{ 1
} }
& {\small Description} &
\begin{minipage}[t]{13cm}{\scriptsize
Transfer data from Data Facility to US and Chilean DACs over an
uninterrupted 1 week period.\\[2\baselineskip]

\vspace{\dp0}
} \end{minipage} \\ \cdashline{2-3}
& {\small Test Data} &
\begin{minipage}[t]{13cm}{\scriptsize
Data Release

\vspace{\dp0}
} \end{minipage}
\\ \cdashline{2-3}
& {\small Expected Result} &
\begin{minipage}[t]{13cm}{\scriptsize
Data transfers without significant failures or extended latency spikes

\vspace{\dp0}
} \end{minipage}
\\ \hdashline
        \\ \midrule
            \multirow{3}{*}{\parbox{1.3cm}{ 2
} }
& {\small Description} &
\begin{minipage}[t]{13cm}{\scriptsize
Analyze network logs and compare with historical data on the links.

\vspace{\dp0}
} \end{minipage} \\ \cdashline{2-3}
& {\small Test Data} &
\begin{minipage}[t]{13cm}{\scriptsize
NA

\vspace{\dp0}
} \end{minipage}
\\ \cdashline{2-3}
& {\small Expected Result} &
\begin{minipage}[t]{13cm}{\scriptsize
The networks can transfer data at archToDacBandwidth = 10000{[}megabit
per second{]}, i.e. at or exceeding rates specified in LDM-142.

\vspace{\dp0}
} \end{minipage}
\\ \hdashline
        \\ \midrule
    \end{longtable}


\subsection{[LVV-89] DMS-REQ-0189-V-01: Archive to Data Access Center Network Availability }\label{lvv-89}

\begin{longtable}{ccccc}
\hline
\textbf{Jira Link} & \textbf{Assignee} & \textbf{Status} & \textbf{Priority} & \textbf{Test Cases}\\ \hline
\href{https://jira.lsstcorp.org/browse/LVV-89}{LVV-89} &
Robert Gruendl & Not Covered & 1a &
\begin{tabular}{c}
LVV-T201 \\
\end{tabular}
\\
\hline
\end{longtable}

\textbf{Verification Element Description:} \\
This requirement needs the network link to be active for a calculated
amount of time (at least 1 week) without failure. This will require
modeling as failures are rare, so an annual MTBF will be estimated from
test results.

{\footnotesize
\begin{longtable}{p{3cm}p{13cm}}
\hline
\multicolumn{2}{c}{\textbf{Upstream Requirements}}\\ \hline
Requirement ID & DMS-REQ-0189 \\ \cdashline{1-2}
Requirement Description & \textbf{Specification:} The Archive to Data Access Center communications
shall be highly available, with MTBF \textgreater{}
\textbf{archToDacNetMTBF}. \\ \cdashline{1-2}
Requirement Parameters & \textbf{archToDacNetMTBF = 180{{[}day{]}}} Mean Time Between Failures
for data service between Archive and DACs, averaged over a one-year
period. \\ \cdashline{1-2}
Requirement Priority & 1b \\ \cdashline{1-2}
Upper Level Requirement &
\begin{tabular}{cl}
DMS-REQ-0161 & Optimization of Cost, Reliability and Availability in Order \\
\end{tabular}
\\ \hline
\end{longtable}
}

\subsubsection{[LVV-T201] Verify implementation of Archive to Data Access Center Network
Availability }

\begin{longtable}{cccccc}
\hline
\multicolumn{6}{c}{\textbf{Test Case Summary}} \\ \hline
\textbf{Jira Link} & \textbf{Owner} & \textbf{Status} & \textbf{Version} & \textbf{Critical Event} &
\textbf{Verification Type} \\ \hline
\href{https://jira.lsstcorp.org/secure/Tests.jspa\#/testCase/LVV-T201}{LVV-T201} &
Jeff Kantor & Draft & 1 & false & Test
\\ \hline
\end{longtable}

\textbf{Objective:} \\
From test and historical data, extrapolate to 1 year to estimate
networks meet archToDacNetMTBF = 180{[}day{]}.

\textbf{Precondition:} \\
Data is staged in LDF and data transfer capabilities to US DAC and
Chilean DAC are in place.\\
At least 6 months of historical monitoring data is available on these
network links.\\
As-built documentation for all of the above is available.

\textbf{Predecessors:} \\
PMCS DMTC-8100-2550 Complete

\textbf{Test Personnel:} \\
Ron Lambert (LSST), Albert Astudillo (REUNA), Jeronimo Bezerra
(FIU/AmLight), Matt Kollross (NCSA)

\textbf{Test Procedure}
    \begin{longtable}[]{p{1.3cm}p{2cm}p{13cm}}
    %\toprule
    Step & \multicolumn{2}{@{}l}{Description, Input Data and Expected Result} \\ \toprule
    \endhead
            \multirow{3}{*}{\parbox{1.3cm}{ 1
} }
& {\small Description} &
\begin{minipage}[t]{13cm}{\scriptsize
Transfer data between archive and DACs over uninterrupted 1 week period.

\vspace{\dp0}
} \end{minipage} \\ \cdashline{2-3}
& {\small Test Data} &
\begin{minipage}[t]{13cm}{\scriptsize
Data Release or petabyte-scale test data set

\vspace{\dp0}
} \end{minipage}
\\ \cdashline{2-3}
& {\small Expected Result} &
\begin{minipage}[t]{13cm}{\scriptsize
Data transfers without failures or extended latency spikes

\vspace{\dp0}
} \end{minipage}
\\ \hdashline
        \\ \midrule
            \multirow{3}{*}{\parbox{1.3cm}{ 2
} }
& {\small Description} &
\begin{minipage}[t]{13cm}{\scriptsize
Analyze test data and compare to historical data. Extrapolate to 1 year
testimate of MTBF.

\vspace{\dp0}
} \end{minipage} \\ \cdashline{2-3}
& {\small Test Data} &
\begin{minipage}[t]{13cm}{\scriptsize
NA

\vspace{\dp0}
} \end{minipage}
\\ \cdashline{2-3}
& {\small Expected Result} &
\begin{minipage}[t]{13cm}{\scriptsize
Networks can meet archToDacNetMTBF = 180{[}day{]} at or exceeding rates
specified in LDM-142.

\vspace{\dp0}
} \end{minipage}
\\ \hdashline
        \\ \midrule
    \end{longtable}


\subsection{[LVV-90] DMS-REQ-0190-V-01: Archive to Data Access Center Network Reliability }\label{lvv-90}

\begin{longtable}{ccccc}
\hline
\textbf{Jira Link} & \textbf{Assignee} & \textbf{Status} & \textbf{Priority} & \textbf{Test Cases}\\ \hline
\href{https://jira.lsstcorp.org/browse/LVV-90}{LVV-90} &
Robert Gruendl & Not Covered & 1a &
\begin{tabular}{c}
LVV-T202 \\
\end{tabular}
\\
\hline
\end{longtable}

\textbf{Verification Element Description:} \\
This requirement is verified by reconnecting and recovering transfer of
data between archive and DACs, after disconnecting fiber at an
intermediate location between archive and DACs.

{\footnotesize
\begin{longtable}{p{3cm}p{13cm}}
\hline
\multicolumn{2}{c}{\textbf{Upstream Requirements}}\\ \hline
Requirement ID & DMS-REQ-0190 \\ \cdashline{1-2}
Requirement Description & \textbf{Specification:} The Archive to Data Access Center communications
shall be highly reliable, with MTTR \textless{}
\textbf{archToDacNetMTTR}. \\ \cdashline{1-2}
Requirement Parameters & \textbf{archToDacNetMTTR = 48{{[}hour{]}}} Mean time to repair, measured
over a 1-yr period. \\ \cdashline{1-2}
Requirement Priority & 1b \\ \cdashline{1-2}
Upper Level Requirement &
\begin{tabular}{cl}
DMS-REQ-0161 & Optimization of Cost, Reliability and Availability in Order \\
\end{tabular}
\\ \hline
\end{longtable}
}

\subsubsection{[LVV-T202] Verify implementation of Archive to Data Access Center Network
Reliability }

\begin{longtable}{cccccc}
\hline
\multicolumn{6}{c}{\textbf{Test Case Summary}} \\ \hline
\textbf{Jira Link} & \textbf{Owner} & \textbf{Status} & \textbf{Version} & \textbf{Critical Event} &
\textbf{Verification Type} \\ \hline
\href{https://jira.lsstcorp.org/secure/Tests.jspa\#/testCase/LVV-T202}{LVV-T202} &
Jeff Kantor & Draft & 1 & false & Test
\\ \hline
\end{longtable}

\textbf{Objective:} \\
After disconnecting fiber locations on the Archive to DACs primary path,
verify fail-over to secondary path and capacity on secondary path.
~After reconnecting primary path, recovery of transfer of data on
primary path, and verify recovery can meet chToDacNetMTTR =
48{[}hour{]}.

\textbf{Precondition:} \\
Data is staged in LDF and data transfer capabilities to US DAC and
Chilean DAC are in place.\\
As-built documentation for all of the above is available.

\textbf{Predecessors:} \\
PMCS DMTC-8100-2550 Complete

\textbf{Test Personnel:} \\
Ron Lambert (LSST), Albert Astudillo (REUNA), Jeronimo Bezerra
(FIU/AmLight), Matt Kollross (NCSA)

\textbf{Test Procedure}
    \begin{longtable}[]{p{1.3cm}p{2cm}p{13cm}}
    %\toprule
    Step & \multicolumn{2}{@{}l}{Description, Input Data and Expected Result} \\ \toprule
    \endhead
            \multirow{3}{*}{\parbox{1.3cm}{ 1
} }
& {\small Description} &
\begin{minipage}[t]{13cm}{\scriptsize
Simulate failure on primary paths by disconnecting fiber at an endpoint
location in the archive on the Archive - ~DACs network.

\vspace{\dp0}
} \end{minipage} \\ \cdashline{2-3}
& {\small Test Data} &
\begin{minipage}[t]{13cm}{\scriptsize
NA

\vspace{\dp0}
} \end{minipage}
\\ \cdashline{2-3}
& {\small Expected Result} &
\begin{minipage}[t]{13cm}{\scriptsize
Networks fail over to secondary paths.

\vspace{\dp0}
} \end{minipage}
\\ \hdashline
        \\ \midrule
            \multirow{3}{*}{\parbox{1.3cm}{ 2
} }
& {\small Description} &
\begin{minipage}[t]{13cm}{\scriptsize
Monitor transfers on secondary paths for 1 day.

\vspace{\dp0}
} \end{minipage} \\ \cdashline{2-3}
& {\small Test Data} &
\\ \cdashline{2-3}
& {\small Expected Result} &
\begin{minipage}[t]{13cm}{\scriptsize
Transfers occur without extended failures or extended latency spikes.
~Data transfers on secondary at rates at or above those specified in
LDM-142.

\vspace{\dp0}
} \end{minipage}
\\ \hdashline
        \\ \midrule
            \multirow{3}{*}{\parbox{1.3cm}{ 3
} }
& {\small Description} &
\begin{minipage}[t]{13cm}{\scriptsize
Simulate repair and recovery period by leaving primary fiber
disconnected for at least 1 day, then reconnecting primary fiber.

\vspace{\dp0}
} \end{minipage} \\ \cdashline{2-3}
& {\small Test Data} &
\begin{minipage}[t]{13cm}{\scriptsize
NA

\vspace{\dp0}
} \end{minipage}
\\ \cdashline{2-3}
& {\small Expected Result} &
\begin{minipage}[t]{13cm}{\scriptsize
Wall clock advances by 1 day. ~Network recovers to primary path. ~Verify
entire process meets chToDacNetMTTR = 48{[}hour{]}.

\vspace{\dp0}
} \end{minipage}
\\ \hdashline
        \\ \midrule
            \multirow{3}{*}{\parbox{1.3cm}{ 4
} }
& {\small Description} &
\begin{minipage}[t]{13cm}{\scriptsize

\vspace{\dp0}
} \end{minipage} \\ \cdashline{2-3}
& {\small Test Data} &
\\ \cdashline{2-3}
& {\small Expected Result} &
\\ \hdashline
        \\ \midrule
    \end{longtable}


\subsection{[LVV-91] DMS-REQ-0191-V-01: Archive to Data Access Center Network Secondary Link }\label{lvv-91}

\begin{longtable}{ccccc}
\hline
\textbf{Jira Link} & \textbf{Assignee} & \textbf{Status} & \textbf{Priority} & \textbf{Test Cases}\\ \hline
\href{https://jira.lsstcorp.org/browse/LVV-91}{LVV-91} &
Robert Gruendl & Not Covered & 1a &
\begin{tabular}{c}
LVV-T203 \\
\end{tabular}
\\
\hline
\end{longtable}

\textbf{Verification Element Description:} \\
This requirement is verified by reconnecting and recovering transfer of
data between archive and DACs, after disconnecting fiber at an
intermediate location between archive and DACs.

{\footnotesize
\begin{longtable}{p{3cm}p{13cm}}
\hline
\multicolumn{2}{c}{\textbf{Upstream Requirements}}\\ \hline
Requirement ID & DMS-REQ-0191 \\ \cdashline{1-2}
Requirement Description & \textbf{Specification:} The Archive to Data Access Center communications
shall provide secondary link or transport mechanism (e.g. protected
circuit) for operations support and ``catch up'' in the event of
extended outage. \\ \cdashline{1-2}
Requirement Priority & 1b \\ \cdashline{1-2}
Upper Level Requirement &
\begin{tabular}{cl}
DMS-REQ-0189 & Archive to Data Access Center Network Availability \\
DMS-REQ-0190 & Archive to Data Access Center Network Reliability \\
\end{tabular}
\\ \hline
\end{longtable}
}

\subsubsection{[LVV-T203] Verify implementation of Archive to Data Access Center Network Secondary
Link }

\begin{longtable}{cccccc}
\hline
\multicolumn{6}{c}{\textbf{Test Case Summary}} \\ \hline
\textbf{Jira Link} & \textbf{Owner} & \textbf{Status} & \textbf{Version} & \textbf{Critical Event} &
\textbf{Verification Type} \\ \hline
\href{https://jira.lsstcorp.org/secure/Tests.jspa\#/testCase/LVV-T203}{LVV-T203} &
Kian-Tat Lim & Draft & 1 & false & Test
\\ \hline
\end{longtable}

\textbf{Objective:} \\


\textbf{Precondition:} \\
Data is staged in LDF and data transfer capabilities to US DAC and
Chilean DAC are in place.\\
As-built documentation for all of the above is available.

\textbf{Predecessors:} \\
PMCS DMTC-8100-2550 Complete

\textbf{Test Personnel:} \\
Ron Lambert (LSST), Albert Astudillo (REUNA), Jeronimo Bezerra
(FIU/AmLight), Matt Kollross (NCSA)

\textbf{Test Procedure}
    \begin{longtable}[]{p{1.3cm}p{2cm}p{13cm}}
    %\toprule
    Step & \multicolumn{2}{@{}l}{Description, Input Data and Expected Result} \\ \toprule
    \endhead
            \multirow{3}{*}{\parbox{1.3cm}{ 1
} }
& {\small Description} &
\begin{minipage}[t]{13cm}{\scriptsize
Transfer data between Archive and DACs on primary path over
uninterrupted 1 week period.

\vspace{\dp0}
} \end{minipage} \\ \cdashline{2-3}
& {\small Test Data} &
\begin{minipage}[t]{13cm}{\scriptsize
Data Release or other petabyte-scale test data set.

\vspace{\dp0}
} \end{minipage}
\\ \cdashline{2-3}
& {\small Expected Result} &
\begin{minipage}[t]{13cm}{\scriptsize
Data transfers without failures or extended latency spikes, at or
exceeding rates specified in LDM-142 throughout fail-over period.

\vspace{\dp0}
} \end{minipage}
\\ \hdashline
        \\ \midrule
            \multirow{3}{*}{\parbox{1.3cm}{ 2
} }
& {\small Description} &
\begin{minipage}[t]{13cm}{\scriptsize
Simulate outage on primary path by disconnecting fiber on primary on
Archive side of Archive - DACs networks.

\vspace{\dp0}
} \end{minipage} \\ \cdashline{2-3}
& {\small Test Data} &
\begin{minipage}[t]{13cm}{\scriptsize
NA

\vspace{\dp0}
} \end{minipage}
\\ \cdashline{2-3}
& {\small Expected Result} &
\begin{minipage}[t]{13cm}{\scriptsize
Network fails over to secondary links in \textless{}=
60s.\\[2\baselineskip]

\vspace{\dp0}
} \end{minipage}
\\ \hdashline
        \\ \midrule
            \multirow{3}{*}{\parbox{1.3cm}{ 3
} }
& {\small Description} &
\begin{minipage}[t]{13cm}{\scriptsize
Transfer data between base and archive over secondary equipment
uninterrupted 1 day period.

\vspace{\dp0}
} \end{minipage} \\ \cdashline{2-3}
& {\small Test Data} &
\begin{minipage}[t]{13cm}{\scriptsize
Data Release or other petabyte-scale test data set.

\vspace{\dp0}
} \end{minipage}
\\ \cdashline{2-3}
& {\small Expected Result} &
\begin{minipage}[t]{13cm}{\scriptsize
Data transfers without failures or extended latency spikes, ~at or
exceeding rates specified in LDM-142 throughout fail-over period.

\vspace{\dp0}
} \end{minipage}
\\ \hdashline
        \\ \midrule
            \multirow{3}{*}{\parbox{1.3cm}{ 4
} }
& {\small Description} &
\begin{minipage}[t]{13cm}{\scriptsize
Restore connection on primary link (reconnect fiber).

\vspace{\dp0}
} \end{minipage} \\ \cdashline{2-3}
& {\small Test Data} &
\begin{minipage}[t]{13cm}{\scriptsize
NA

\vspace{\dp0}
} \end{minipage}
\\ \cdashline{2-3}
& {\small Expected Result} &
\begin{minipage}[t]{13cm}{\scriptsize
Network recovers to primary in \textless{}= 60s.

\vspace{\dp0}
} \end{minipage}
\\ \hdashline
        \\ \midrule
    \end{longtable}


\subsection{[LVV-183] DMS-REQ-0352-V-01: Base Wireless LAN (WiFi) }\label{lvv-183}

\begin{longtable}{ccccc}
\hline
\textbf{Jira Link} & \textbf{Assignee} & \textbf{Status} & \textbf{Priority} & \textbf{Test Cases}\\ \hline
\href{https://jira.lsstcorp.org/browse/LVV-183}{LVV-183} &
Robert Gruendl & Not Covered & 1a &
\begin{tabular}{c}
LVV-T192 \\
\end{tabular}
\\
\hline
\end{longtable}

\textbf{Verification Element Description:} \\
At Base Facility, connect to WiFi, test connection speed, i.e. send
email, browse web, and retrieve files from the Internet.

{\footnotesize
\begin{longtable}{p{3cm}p{13cm}}
\hline
\multicolumn{2}{c}{\textbf{Upstream Requirements}}\\ \hline
Requirement ID & DMS-REQ-0352 \\ \cdashline{1-2}
Requirement Description & \textbf{Specification:} The Base LAN shall provide \textbf{minBaseWiFi}
Wireless LAN (WiFi) and Wireless Access Points in the Base Facility to
support connectivity of individual user's computers to the network
backbones. \\ \cdashline{1-2}
Requirement Parameters & \textbf{minBaseWifi = 1000{{[}megabit per second{]}}} Maximum allowable
outage of active DM production. \\ \cdashline{1-2}
Requirement Priority & 2 \\ \cdashline{1-2}
Upper Level Requirement &
\begin{tabular}{cl}
OSS-REQ-0003 & The Base Facility \\
\end{tabular}
\\ \hline
\end{longtable}
}

\subsubsection{[LVV-T192] Verify implementation of Base Wireless LAN (WiFi) }

\begin{longtable}{cccccc}
\hline
\multicolumn{6}{c}{\textbf{Test Case Summary}} \\ \hline
\textbf{Jira Link} & \textbf{Owner} & \textbf{Status} & \textbf{Version} & \textbf{Critical Event} &
\textbf{Verification Type} \\ \hline
\href{https://jira.lsstcorp.org/secure/Tests.jspa\#/testCase/LVV-T192}{LVV-T192} &
Jeff Kantor & Draft & 1 & false & Test
\\ \hline
\end{longtable}

\textbf{Objective:} \\
Verify as-built wireless network at the Base Facility supports
minBaseWiFi bandwidth (1000 Mbs).

\textbf{Precondition:} \\
Base Wireless LAN is installed/configured and Test Personnel have
accounts for email, internet access.\\
As-built documentation for all of the above is available.

\textbf{Predecessors:} \\
PMCS DLP-465 Complete.

\textbf{Test Personnel:} \\
Heinrich Reinking (LSST)

\textbf{Test Procedure}
    \begin{longtable}[]{p{1.3cm}p{2cm}p{13cm}}
    %\toprule
    Step & \multicolumn{2}{@{}l}{Description, Input Data and Expected Result} \\ \toprule
    \endhead
            \multirow{3}{*}{\parbox{1.3cm}{ 1
} }
& {\small Description} &
\begin{minipage}[t]{13cm}{\scriptsize
Test internet web browsing and file download, email at summit and base
over wireless.

\vspace{\dp0}
} \end{minipage} \\ \cdashline{2-3}
& {\small Test Data} &
\begin{minipage}[t]{13cm}{\scriptsize
NA

\vspace{\dp0}
} \end{minipage}
\\ \cdashline{2-3}
& {\small Expected Result} &
\begin{minipage}[t]{13cm}{\scriptsize
Verify as-built wireless network at the Base Facility supports
minBaseWiFi bandwidth (1000 Mbs). Verify wireless signal strength meets
or exceeds typical, and average and peak bandwidths meet or exceed
minBaseWiFI bandwidth.

\vspace{\dp0}
} \end{minipage}
\\ \hdashline
        \\ \midrule
    \end{longtable}


\subsection{[LVV-18491] DMS-REQ-0352-V-02: Base Voice Over IP (VOIP) }\label{lvv-18491}

\begin{longtable}{ccccc}
\hline
\textbf{Jira Link} & \textbf{Assignee} & \textbf{Status} & \textbf{Priority} & \textbf{Test Cases}\\ \hline
\href{https://jira.lsstcorp.org/browse/LVV-18491}{LVV-18491} &
Robert Gruendl & Not Covered & 2 &
\begin{tabular}{c}
LVV-T181 \\
\end{tabular}
\\
\hline
\end{longtable}

\textbf{Verification Element Description:} \\
Verify (a) plannned and (b) as-built VOIP at the Base Facility is
operational and performs as expected (i.e. sufficient number of
extensions allocated properly, no frequent drop-outs, no frequent
jaggies on video, etc.). Test voice calls and videoconferening.

{\footnotesize
\begin{longtable}{p{3cm}p{13cm}}
\hline
\multicolumn{2}{c}{\textbf{Upstream Requirements}}\\ \hline
Requirement ID & DMS-REQ-0352 \\ \cdashline{1-2}
Requirement Description & \textbf{Specification:} The Base LAN shall provide \textbf{minBaseWiFi}
Wireless LAN (WiFi) and Wireless Access Points in the Base Facility to
support connectivity of individual user's computers to the network
backbones. \\ \cdashline{1-2}
Requirement Parameters & \textbf{minBaseWifi = 1000{{[}megabit per second{]}}} Maximum allowable
outage of active DM production. \\ \cdashline{1-2}
Requirement Priority & 2 \\ \cdashline{1-2}
Upper Level Requirement &
\begin{tabular}{cl}
OSS-REQ-0003 & The Base Facility \\
\end{tabular}
\\ \hline
\end{longtable}
}

\subsubsection{[LVV-T181] Verify Base Voice Over IP (VOIP) }

\begin{longtable}{cccccc}
\hline
\multicolumn{6}{c}{\textbf{Test Case Summary}} \\ \hline
\textbf{Jira Link} & \textbf{Owner} & \textbf{Status} & \textbf{Version} & \textbf{Critical Event} &
\textbf{Verification Type} \\ \hline
\href{https://jira.lsstcorp.org/secure/Tests.jspa\#/testCase/LVV-T181}{LVV-T181} &
Jeff Kantor & Draft & 1 & false & Test
\\ \hline
\end{longtable}

\textbf{Objective:} \\
Verify as-built VOIP at the Base Facility is operational and performs as
expected (i.e. sufficient number of extensions allocated properly, no
frequent drop-outs, no frequent jaggies on video, etc.) on both voice
calls and videoconferening.

\textbf{Precondition:} \\
Base VOIP is installed/configured and Test Personnel have phone sets.
~Base Videoconference system is installed/configured. ~Summit,
Headquarters, and/or LDF Videconference system is
installed/configured.\\
As-built documentation for all of the above is available.

\textbf{Predecessors:} \\
PMCS DLP-465 Complete\\
PMCS IT-702 Complete

\textbf{Test Personnel:} \\
Heinrich Reinking (LSST), another LSST DM Person at Summit,
Headquarters, or LDF

\textbf{Test Procedure}
    \begin{longtable}[]{p{1.3cm}p{2cm}p{13cm}}
    %\toprule
    Step & \multicolumn{2}{@{}l}{Description, Input Data and Expected Result} \\ \toprule
    \endhead
            \multirow{3}{*}{\parbox{1.3cm}{ 1
} }
& {\small Description} &
\begin{minipage}[t]{13cm}{\scriptsize
Test voice calls over VOIP system from Base Facility to locations in
~Base and to other Rubin Observatory facilities.

\vspace{\dp0}
} \end{minipage} \\ \cdashline{2-3}
& {\small Test Data} &
\\ \cdashline{2-3}
& {\small Expected Result} &
\begin{minipage}[t]{13cm}{\scriptsize
As-built VOIP at the Base Facility is operational and performs as
expected (i.e. sufficient number of extensions allocated properly, no
frequent drop-outs, etc.).

\vspace{\dp0}
} \end{minipage}
\\ \hdashline
        \\ \midrule
            \multirow{3}{*}{\parbox{1.3cm}{ 2
} }
& {\small Description} &
\begin{minipage}[t]{13cm}{\scriptsize
Test video conferences over ~system from Base Facility to locations in
Base and to other Rubin Observatory facilities.

\vspace{\dp0}
} \end{minipage} \\ \cdashline{2-3}
& {\small Test Data} &
\\ \cdashline{2-3}
& {\small Expected Result} &
\begin{minipage}[t]{13cm}{\scriptsize
Verify (a) plannned and (b) as-built VOIP at the Base Facility is
operational and performs as expected (i.e. no frequent drop-outs, no
frequent audio glitches, no frequent jaggies on video, etc.).

\vspace{\dp0}
} \end{minipage}
\\ \hdashline
        \\ \midrule
    \end{longtable}



\appendix
\section{References\label{sect:references}}
\renewcommand{\refname}{}
\bibliography{lsst,refs,books,refs_ads,local.bib}

\section{Acronyms \label{sect:acronyms}} % include acronyms.tex generated by the generateAcronyms.py (in texmf/scripts)
\addtocounter{table}{-1}
\begin{longtable}{p{0.145\textwidth}p{0.8\textwidth}}\hline
\textbf{Acronym} & \textbf{Description}  \\\hline

BDC &  Base Data Center \\\hline
BERT & Bit Error Rate Tester \\\hline
CCS & Camera Control System \\\hline
CHI & Chicago \\\hline
CHMPGN & Champaign (Illinois) \\\hline
CISS & Computer Infrastructure Services South (part of the former NOAO Cerro Tololo Inter-american Observatory (CTIO), now merged into NSF’S OIR Lab Central Operating Services) \\\hline
CTIO & Cerro Tololo Inter-American Observatory \\\hline
DAC & Data Access Center \\\hline
DAQ & Data Acquisition System \\\hline
DM & Data Management \\\hline
DMCS & Data Management Control System \\\hline
DMS & Data Management Subsystem \\\hline
DMS-REQ & Data Management top level requirements (\citeds{LSE-61}) \\\hline
DMSR & DM System Requirements; LSE-61 \\\hline
DMSSIT & DM Subsystem Integration Test \\\hline
DMTR & DM Test (Plan and) Report \\\hline
DTN & Data Transfer Node \\\hline
DWDM & Dense Wave Division Multiplex \\\hline
EFD & Engineering and Facility Database \\\hline
EPO & Education and Public Outreach \\\hline
FIU & Florida International University \\\hline
FL & Florida \\\hline
HL & Higher Level \\\hline
IP & Internet Protocol \\\hline
ISO & International Standards Organization \\\hline
IT & Information Technology \\\hline
LAN & Local Area Network \\\hline
LATISS & LSST Atmospheric Transmission Imager and Slitless Spectrograph \\\hline
LDF & LSST Data Facility \\\hline
LDM & LSST Data Management (Document Handle) \\\hline
LHN & Long-Haul Networks \\\hline
LL & Lower Level \\\hline
LS & La Serena \\\hline
LSE & LSST Systems Engineering (Document Handle) \\\hline
LSST & Large Synoptic Survey Telescope \\\hline
LVV & LSST Verification and Validation (Jira project) \\\hline
MTBF & Mean Time Between Failures \\\hline
MTTR & Mean Time to Repair \\\hline
NCSA & National Center for Supercomputing Applications \\\hline
NET & Network Engineering Team \\\hline
OCS & Observatory Control System \\\hline
OSI & Open System Interconnect \\\hline
OSS & Observatory System Specifications; LSE-30 \\\hline
OTDR & Optical Time Domain Reflectometer \\\hline
PMCS & Project Management Controls System \\\hline
REUNA & Red Universitaria Nacional \\\hline
SC & Science Collaboration \\\hline
SCL & Santiago, Chile \\\hline
SIT & LSST System Integration Test \\\hline
SL & Same Level \\\hline
SLAC & SLAC National Accelerator Lab \\\hline
TCS & Telescope Control System \\\hline
US & United States \\\hline
VNVD & Vera C Rubin Observatory Network Verification Document \\\hline
VOIP & Voice Over Internet Protocol \\\hline
WBS & Work Breakdown Structure \\\hline
\end{longtable}


\end{document}
